\documentclass[a4paper, 12pt]{article}
\usepackage[utf8]{inputenc}
\usepackage[russian]{babel}
\usepackage{amsfonts}
\usepackage{amsmath} 
\usepackage{amssymb}
\usepackage{amsthm}
\usepackage{mathtools}
\usepackage{indentfirst} 
\usepackage{graphicx}
\usepackage[justification=centering]{caption}
\usepackage{float}

\usepackage{geometry} % Меняем поля страницы
\geometry{left=2cm}% левое поле
\geometry{right=1.5cm}% правое поле
\geometry{top=1.5cm}% верхнее поле
\geometry{bottom=3cm}% нижнее поле

\graphicspath{{./img/}}

\newtheoremstyle{mythm}  % follow `plain` defaults but change HEADSPACE.
  	{\topsep}% measure of space to leave above the theorem. E.g.: 3pt
	{\topsep}% measure of space to leave below the theorem. E.g.: 3pt
	{\itshape}% name of font to use in the body of the theorem
	{0pt}% measure of space to indent
	{\bfseries}% name of head font
	{}% punctuation between head and body
	{\newline}% space after theorem head; " " = normal interword space
	{\thmname{#1}\thmnumber{ #2.}\thmnote{ #3}}% Manually specify head

\theoremstyle{mythm}
\newtheorem{theorem}{Теорема}

\begin{document}
	% Титульный лист
	\begin{titlepage}
		\begin{center}
			Санкт-Петербургский политехнический университет Петра Великого \\ Физико-механический институт \\ Высшая школа прикладной математики и вычислительной физики
		\end{center}
		\vspace{10em}
		\begin{center}
			\Large Курсовая работа \\ по дисциплине "Численные методы"
		\end{center}
		\vspace{1em}
		\begin{center}
			\Huge Сравнение решения задачи Коши для ОДУ 1 порядка модифицированным методом Эйлера и явным методом Адамса 2 порядка.
		\end{center}
		\vspace{15em}
		{\Large 
			
			Выполнил: \hfill студент гр. 5030102/00003 Красников Р.А.
			\vspace{1em}
			
			Преподаватель: \hfill Добрецова С.Б.}
		\vspace{\fill}
		\begin{center}
			Санкт-Петербург \\ 2022
		\end{center}
	\end{titlepage}
	\newpage
	
	\tableofcontents
	
	\newpage
	
	\section{Задание.}
	
	Дано обыкновенное дифференциальное уравнение 1 порядка на отрезке. Поставить задачу Коши и найти ее решение
	\begin{enumerate}
		\item модифицированным методом Эйлера
		\item явным методом Адамса 2 порядка
	\end{enumerate}
	
	Провести сравнение результатов по графикам ошибок на отрезке для двух значений шага. Исследовать зависимость нормы погрешности от возмущения начальных условий.
	
	\section{Постановка задачи.}
	
	Дано обыкновенное дифференциальное уравнение 1 порядка
	\begin{equation}
		F(x,y,y')=0,\ x\in[a,b]
	\end{equation}
	разрешимое относительно старшей производной, то есть оно может быть записано в виде
	\begin{equation} \label{eq}
		y'=f(x,y),\ x\in[a,b]
	\end{equation}
	Также задано начальное условие
	\begin{equation} \label{cond}
		y(a)=y_0
	\end{equation}
	Известно, что существует единственная функция $\varphi(x):[a,b]\rightarrow\mathbb{R}$ -- решение задачи Коши \eqref{eq}, \eqref{cond}, т.е.
	\begin{equation}
		\begin{gathered}
			\forall x\in[a,b]: \varphi'(x)=f(x,\varphi(x)),\\
			\varphi(a)=y_0
		\end{gathered}
	\end{equation}

	Для численного решения задачи также задана сетка $\{x_i\}_{i=0}^n$.
	
	Необходимо найти сеточную функцию $\{x_i,y_i\}_{i=0}^n$ такую, что
	\begin{equation}
		\forall i=\overline{0,n}: y_i\approx\varphi(x_i)
	\end{equation}

	\section{Предварительный анализ задачи.}
	
	Пусть поставлена задача Коши 
	\begin{equation} \label{eq:cauchy_task}
		\begin{cases}
			y'=f(x,y),\ x\in[a,b],\\
			y(a)=y_0
		\end{cases}
	\end{equation}
	
	\subsection{Переход к модифицированному методу Эйлера.}
	
	Зададимся целью построить явный одношаговый метод 2 порядка:
	\begin{equation} \label{scheme}
		y_{k+1}=\Phi_f(h,x_{k},y_{k})
	\end{equation}
	В предположении достаточной гладкости функции $y(x)$ запишем ее разложение по формуле Тейлора в окрестности точки $x$, удерживая слагаемые до $o(h^2)$, поскольку строится метод 2 порядка
	\begin{equation} \label{taylor}
		y(x+h)=y(x)+\underbrace{\dfrac{h}{1!}y'(x)+\dfrac{h^2}{2!}y''(x)}_{\Delta y(x)}+o(h^2)
	\end{equation} 

	Идея методов Рунге-Кутты, к которым относится и модифицированный метод Эйлера, состоит в отыскании функции $\delta y(x)$, отличающейся от $\Delta y(x)$ на бесконечно малую нужного порядка малости, т.е.
	\begin{equation}
		\delta y(x) = \Delta y(x) + o(h^2)
	\end{equation}
	Если удастся найти $\delta y(x)$ в виде
	\begin{equation}
		\delta y(x) = h\sum\limits_{i=1}^l \rho_if(x+\delta x_i, y + \delta y_i)
	\end{equation}
	то отбрасыванием $o(h^2)$ из \eqref{taylor} получим метод 2 порядка, удовлетворяющий виду \eqref{scheme}
	\begin{equation}
		y_{k+1} = y_{k} + \delta y(x_{k})
	\end{equation}
	
	Обычно, при построении методов Рунге-Кутты, функция $\delta y(x)$ ищется в виде
	\begin{equation}
		\begin{gathered}
			\delta y(x) = h\sum\limits_{i=1}^l\rho_iK_i,\\
			\begin{dcases}
				K_1 = f(x,y),\\
				K_2 = f(x + \alpha_2h, y + \beta_{21}K_1),\\
				K_3 = f(x + \alpha_3h, y + \beta_{31}K_1 + \beta_{32}K_2),\\
				...\\
				K_l = f(x + \alpha_lh, y + \sum\limits_{j=1}^{l-1}\beta_{ji}K_i)
			\end{dcases}
		\end{gathered}		
	\end{equation}

	При непосредственном построении модифицированного метода Эйлера получаем следующие значения:
	\begin{equation}
		l=2,\ \rho_1=0,\ \rho_2=1,\ \alpha_2=\beta_{21}=\dfrac{1}{2}
	\end{equation}

	\subsection{Переход к явному методу Адамса 2 порядка.}
	
	Проинтегрируем уравнение $y'=f(x,y)$ по отрезку $[x_{k-1}, x_k]$:
	\begin{equation} \label{int_adams}
		y_k=y_{k-1}+\int\limits_{x_{k-1}}^{x_k}f(x,y(x))dx=y_{k-1}+\int\limits_{x_{k-1}}^{x_k}F(x)dx
	\end{equation}
	Будем строить $r$-шаговый метод.
	
	Аппроксимируем функцию $F(x)$ интерполяционным полиномом Лагранжа $L_m(x)$ на сеточной функции $\{\overline{x}_i, F_i\}_{i=0}^m$, где $F_i= F(\overline{x}_i)$.
	\begin{equation}
		F(x) = L_m(x) + R_m(x)
	\end{equation} 
	и подставим в \eqref{int_adams}
	\begin{equation}
		y_k=y_{k-1}+\underbrace{\int\limits_{x_{k-1}}^{x_k}L_m(x)dx}_{I_1} + \underbrace{\int\limits_{x_{k-1}}^{x_k}R_m(x)dx}_{I_2}
	\end{equation}
	Здесь $\{\overline{x}_i\}_{i=0}^m$ -- сетка, такая что $\overline{x}_r=\overline{x}_0+hr=x_0+hk=x_k$. Не для всех $x_k$ такую сетку можно построить, поэтому для методов Адамса требуется вычислять разгонные точки.
	Пренебрежем интегралом $I_2$ от остаточного члена полинома Лагранжа, а в интеграле $I_1$ выполним замену переменной по формуле $x=\overline{x}_0+ht$
	\begin{equation}
		I_1=\int\limits_{\overline{x}_{r-1}}^{\overline{x}_r}L_m(x)dx=h\int\limits_{r-1}^{r}L_m(\overline{x}_0+ht)dt
	\end{equation}
	По формуле для полинома Лагранжа на равномерной сетке
	\begin{equation}
		I_1=h\int\limits_{r-1}^{r}\sum\limits_{j=0}^mF_j\frac{(-1)^{m-j}}{(m-j)!j!}\frac{\overline{\omega}(t)}{t-j}dt,
	\end{equation}
	где $\displaystyle \overline{\omega}(t)=t(t-1)\dots(t-m)=\prod\limits_{j=0}^m(t-j)$. Упростим выражение
	\begin{equation}
		I_1=h\sum\limits_{j=0}^m\underbrace{\bigg[\frac{(-1)^{m-j}}{(m-j)!j!}\int\limits_{r-1}^r\frac{\overline{\omega}(t)}{t-j}dt\bigg]}_{\beta_j}F_j=h\sum\limits_{j=0}^m\beta_{j}F_j
	\end{equation}
	Получили метод
	\begin{equation}
		y_{k}=y_{k-1}+h\sum\limits_{j=0}^m\beta_jf(x_{k-r+j}, y_{k-r+j})
	\end{equation}

	Мы строили явный метод Адамса 2 порядка. Чтобы удовлетворить требованию второго порядка аппроксимации нужно потребовать $m=1$ -- степень полинома Лагранжа. Метод должен быть явным, т.е. в правой части не должно содержаться слагаемого с номером $r$, или, что равносильно, сетка $\{\overline{x}_i\}_{i=0}^m$ не должна включать в себя $x_k$ (т.е. значение функции $F(x)$ в точке $x_k$ не интерполируется, а экстраполируется), поэтому $m=r-1\Rightarrow r=2$ -- явный метод Адамса 2 порядка двухшаговый. 
	
	При непосредственном построении явного метода Адамса 2 порядка получаем следующие значения
	\begin{equation}
		\beta_0 = -\dfrac{1}{2},\ \beta_1 = \dfrac{3}{2}
	\end{equation}

	\section{Алгоритмы методов.}
	
	\subsection{Условия применимости исследуемых методов.}
	
	Из предварительного анализа задачи видно, что если задача Коши \eqref{eq:cauchy_task} имеет единственное решение и построенные по исследуемым методам последовательности сходятся, то их предел есть точное решение задачи Коши. Таким образом условие применимости обоих методов
	\begin{enumerate}
		\item Задача Коши \eqref{eq:cauchy_task} имеет единственное решение
	\end{enumerate}

	В таком виде условие применимости может быть не удобным в применении к конкретной задаче, поэтому можно воспользоваться теоремами
	\begin{theorem}[О единственности решения задачи Коши.]
		Если функция $f(x,y)$ удовлетворяет условию Липшица по y в прямоугольнике $\Pi = [a, b] \times [c, d]$, т.е.
		\begin{equation}
			\exists L > 0\ \forall x \in [a,b]\ \forall y_1,y_2\in[c,d]\ |f(x,y_1) - f(x,y_2)| < L|y_1-y_2|,
		\end{equation} 
		то решение задачи Коши \eqref{eq:cauchy_task} единственно.
	\end{theorem}
	\begin{theorem}[О достаточном условии липшицевости функции $f(x,y)$.]
		Если функция $f(x,y)$ имеет непрерывную частную производную $\dfrac{\partial f}{\partial y}$ в прямоугольнике $\Pi = [a, b] \times [c, d]$, то $f(x,y)$ удовлетворяет условию Липшица по $y$ в прямоугольнике $\Pi$, и для константы Липшица справедливо выражение
		\begin{equation}
			L = \max\limits_{(x,y)\in \Pi}\dfrac{\partial f}{\partial y}\bigg|_{(x,y)}
		\end{equation}
	\end{theorem}
	
	При анализе задачи область значений решения неизвестна. В связи с этим нужно потребовать чтобы условия теорем выполнялись для достаточно больших отрезков $[c,d] \supset E(\varphi)$, где $E(\varphi)$ -- область значений точного решения $\varphi(x)$ задачи Коши \eqref{eq:cauchy_task}. Таким образом, достаточное условие единственности задачи Коши
	\begin{enumerate}
		\item $\dfrac{\partial f}{\partial y}$ непрерывна
	\end{enumerate}
	
	\subsection{Этапы решения поставленной задачи Коши исследуемыми методами.}
	
	В соответствии с постановкой задачи, дана задача Коши для дифференциального уравнения, не разрешенного, но разрешимого, относительно производной. Поэтому первый шаг состоит в разрешении дифференциального уравнения относительно производной.
	
	После разрешения уравнения относительно производной, получаем задачу Коши в виде \eqref{eq:cauchy_task} и к ней уже непосредственно применяем исследуемые методы. 
	
	Таким образом, этапы решения задачи следующие:
	\begin{enumerate}
		\item Разрешить дифференциальное уравнение относительно производной
		\item Применить численный метод решения задачи Коши для дифференциального уравнения 1 порядка, разрешенного относительно производной
	\end{enumerate}
	
	\subsection{Алгоритмы исследуемых методов.}
	
	\subsubsection{Модифицированный метод Ньютона.}
	
	\begin{enumerate}
		\item Ввести функцию $f(x,y)$, начальное условие $y_0$, отрезок $[a,b]$, число отрезков разбиения $m$ (число точек равномерной сетки составит $m+1$).
		\item Для всех $i=\overline{1,m}$ вычислить
		\begin{equation} \label{euler_method}
			y_i = y_{i-1} + hf(x_{i-1} + \frac{h}{2}, y_{i-1}+\frac{h}{2}f(x_{i-1},y_{i-1}))
		\end{equation}
		где $h=\dfrac{b-a}{m}$, $x_i=a+ih$.
		\item $\{(x_i,y_i)\}_{i=0}^m$ -- искомая сеточная функция
	\end{enumerate}

	\subsubsection{Явный метод Адамса 2 порядка.}
	
	\begin{enumerate}
		\item Ввести функцию $f(x,y)$, начальное условие $y_0$, отрезок $[a,b]$, число отрезков разбиения $m$ (число точек равномерной сетки составит $m+1$).
		\item Вычислить разгонную точку (с помощью модифицированного метода Эйлера):
		\begin{equation} \label{euler_boost}
			y_1 = y_0 + hf(x_0 + \frac{h}{2}, y_0+\frac{h}{2}f(x_0,y_0)),
		\end{equation}
		где $h=\dfrac{b-a}{m}$, $x_i=a+ih$.
		\item Для всех $i=\overline{2,m}$ вычислить
		\begin{equation} \label{adams_method}
			y_i = y_{i-1} + \frac{h}{2}(3f(x_{i-1},y_{i-1})-f(x_{i-2}, y_{i-2}))
		\end{equation}
		\item $\{(x_i,y_i)\}_{i=0}^m$ -- искомая сеточная функция
	\end{enumerate}
	
	\subsection{Теоретическая оценка результатов исследования.}
	
	\subsubsection{Исследование полученных решений и их погрешностей.}
	
	Из алгоритмов исследуемых методов видно, что при решении задачи Коши модифицированным методом Эйлера нужно на каждой итерации дважды вычислять значение функции $f(x,y)$, в то время как при решении явным методом Адамса 2 порядка -- только один раз (кроме итерации для вычисления разгонной точки и первой итерации вычисления непосредственно по формуле явного метода Адамса \eqref{adams_method}).
	
	В связи с этим ожидается, что решение полученное явным методом Адамса 2 порядка будет не точнее решения, полученного модифицированным методом Эйлера для одного значения шага, ведь в противном случае получится, что вычислений по методу Адамса нужно меньше и решение при этом точнее, что поставило бы под вопрос эффективность модифицированного метода Эйлера. 
	
	Также оба исследуемых метода имеют 2 порядок аппроксимации, поэтому ожидается, что норма глобальной погрешности при уменьшении шага в 2 раза уменьшится приблизительно в 4 раза.
	
	\subsubsection{Исследование зависимости нормы погрешности от возмущения начального условия.}
	
	Рассмотрим задачу Коши, зависящую от параметра
	\begin{equation} \label{eq:cauchy_param}
		\begin{cases}
			y' = f(x,y,\mu),\\
			y(x_0,\mu) = y_0
		\end{cases}
	\end{equation}
	Из курса математического анализа известна следующая теорема
	\begin{theorem}[О непрерывности решения задачи Коши по параметру.] \label{th:param}
		Если функция $f(x,y,\mu)$ и ее частная производная $\dfrac{\partial f}{\partial y}$ непрерывны в некоторой области $G$ и точка $(x_0, y_0, \mu_0)\in G$, то решение задачи Коши \eqref{eq:cauchy_param} непрерывно в некоторой области $D =\{(x,\mu)\in\mathbb{R}^2\ |\ |x-x_0|<\delta_1, |\mu-\mu_0|<\delta_2\}$
	\end{theorem}

	Теперь в задаче Коши
	\begin{equation}
		\begin{cases} \label{eq:cauchy_task_x0}
			y' = f(x,y),\\
			y(a) = y_0
		\end{cases}
	\end{equation} считаем $y_0$ параметром. Выполним замену переменных по формулам $\widetilde{x} = x - a$, $\widetilde{y}(\widetilde{x}, y_0) = y(x) - y_0 = y(\widetilde{x} + a) - y_0$. Тогда правая часть запишется в виде $f(\widetilde{x} + a, \widetilde{y} + y_0)$.

	В новых переменных имеем задачу Коши, зависящую от параметра $y_9$
	\begin{equation}
		\begin{cases} \label{eq:cauchy_task_tilde}
			y' = f(\widetilde{x} + a, \widetilde{y} + y_0),\\
			\widetilde{y}(0, y_0) = 0
		\end{cases}
	\end{equation}

	Наконец, чтобы в точности удовлетворить условию теоремы \ref{th:param}, можно положить $\widetilde{f}(\widetilde{x}, \widetilde{y}, y_0) = f(\widetilde{x} + a, \widetilde{y} + y_0)$. Из этой теоремы следует, что в условиях достаточной гладкости функции $\widetilde{f}(\widetilde{x}, \widetilde{y}, y_0)$ решение задачи Коши \eqref{eq:cauchy_task_tilde} непрерывно по параметру $y_0$ -- то есть начальному условию задачи Коши \eqref{eq:cauchy_task}.
	
	Это значит, что интегральная кривая, проходящая через точку $(a, y_0 \pm \Delta y_0)$ для небольших $\Delta y_0$ расположена на небольшом расстоянии от кривой, проходящей через точку $(a, y_0)$. Если в начальное условие задачи внесено небольшое возмущение, то интегральная кривая, являющаяся пределом последовательности, построенной каждым из методов, при $h \rightarrow 0$, не сильно отстоит от точного решения.
	
	Таким образом, ожидается, что оба исследуемых метода устойчивы по начальному условию.
	
	\section{Контрольные тесты.}
	
	\subsection{Решаемая задача Коши.}
	
	Исследование проведено для задачи Коши
	\begin{equation} \label{mytask}
		\begin{cases}
			xy'-2x^2\sqrt{y}=4y,\ x\in[1,2],\\
			y(1)=1
		\end{cases}
	\end{equation}
	Ее точное решение $\varphi(x)=x^4(\ln x+1)^2$.
	
	\subsection{Численные эксперименты.}
	
	Графики полученных решений и их ошибок построены для шагов $h_1=0.2$ и $h_2=0.1$.
	
	Для исследования зависимости нормы погрешности от возмущения начальных условий были выбраны значения шага $h=0.1$ и возмущений начального условия $\Delta y_0=(1.1)^{-i}, i=\overline{0,85}$. Были проведены 2 эксперимента: для возмущений вида $y(a)=y_0+\Delta y_0$ и $y(a)=y_0-\Delta y_0$. 
	
	\section{Численный анализ.}
	
	\subsection{Анализ полученных решений и их погрешностей.} \label{analysis:ans}
	
	Полученные в процессе исследования графики решений задачи Коши \eqref{mytask} представлены на рисунке \ref{fig:ans}.
	
	\begin{figure}[H]\centering
		\includegraphics[width=.49\textwidth]{ans_Euler}
		\includegraphics[width=.49\textwidth]{ans_Adams}
		\caption{Решения, полученные модифицированным методом Эйлера (слева) и явным методом Адамса 2 порядка (справа).}\label{fig:ans}
	\end{figure}

	На графиках полученных решений видно, что для всех шагов оба метода дают решение, значения которого в узлах меньше, чем точное. Приближенное решение "не успевает"\ за ростом точного. Графики погрешностей данных решений представлены на рисунке \ref{fig:err}.
	
	\begin{figure}[H]\centering
		\includegraphics[width=.75\textwidth]{err_on_x}
		\caption{Ошибки решений, полученных модифицированным методом Эйлера и явным методом Адамса 2 порядка.}\label{fig:err}
	\end{figure}

	На графиках погрешностей полученных решений видно, что погрешности для всех шагов и методов возрастают быстрее, чем линейно. Модифицированный метод Эйлера и исследованный явный метод Адамса имеют 2 порядок аппроксимации, значит при уменьшении шага в 2 раза (c $0.2$ до $0.1$) ошибка должна была уменьшиться приблизительно в 4 раза. 
	
	Например, в точке $b=2$ видно, что для модифицированного метода Эйлера ошибка уменьшилась приблизительно в $5.4/1.7\approx3.2$ раза, а для явного метода Адамса 2 порядка -- приблизительно в $8.5/3.0\approx2.8$ раза. Это объясняется тем, что выражение для погрешности имеет вид $Ch^2$ только при достаточно малых $h$, а в общем случае имеет вид $Ch^2 + o(h^2)$ -- в проведенном эксперименте $h$ еще достаточно велико и $o(h^2)$ дает ощутимую поправку. На рисунке \ref{fig:err_on_h} можно видеть, что для шага $0.1$ и больше зависимость действительно отклоняется от прямой $y=Ch^2$ вогнутым образом.
	
	\begin{figure}[H]\centering
		\includegraphics[width=.75\textwidth]{err_on_h}
		\caption{Зависимость бесконечной нормы погрешности решения от шага.}\label{fig:err_on_h}
	\end{figure}
	
	Также отметим, что для одних и тех же значений шага модифицированный метод Эйлера дал более точное решение, чем явный метод Адамса 2 порядка. Этот результат вполне ожидаем, поскольку модифицированный метод Эйлера требует двух вычислений значения функции $f(x,y)$ на каждой итерации, в то время как явный метод Адамса 2 порядка -- только одного вычисления (кроме итерации для вычисления разгонной точки и первой итерации вычисления непосредственно по формуле явного метода Адамса \eqref{adams_method}).
	
	\subsection{Анализ зависимости нормы погрешности от возмущения начальных условий.}
	
	Полученные в процессе исследования зависимости нормы погрешности от возмущения начальных условий графики приведены на рисунке \ref{fig:err_on_dy}.
	
	\begin{figure}[H]\centering
		\includegraphics[width=.49\textwidth]{err_on_dyplus}
		\includegraphics[width=.49\textwidth]{err_on_dyminus}
		\caption{Зависимости бесконечной нормы погрешности решения от возмущения начального условия вида $y(1) = 1.0+\Delta y_0$ (слева) и $y(1) = 1.0-\Delta y_0$ (справа).}\label{fig:err_on_dy}
	\end{figure}

	На графиках исследуемой зависимости видно, что для обоих методов имеет место асимптота при $\Delta y_0 \rightarrow 0$, притом равная погрешности соответствующих методов для значения шага $0.1$, приведенной на рисунке \ref{fig:err}. Для $\Delta y_0 \leq 10^{-3}$ отклонение погрешности от асимптоты незначительное. 
	
	Интересно отметить, что для возмущения начального условия вида $y(1)=1.0+\Delta y_0$ для обоих методов наблюдается минимум в окрестности точки $\Delta y_0 = 10^{-1}$, притом для возмущения вида $y(1)=1.0-\Delta y_0$ такого минимума не наблюдается. Это можно объяснить, проанализировав полученные методами решения для двух значений шага, графики которых приведены на рисунке \ref{fig:ans}. На графиках видно, что численное решение всюду меньше, чем точное и, как уже было отмечено в пункте \ref{analysis:ans}, "не успевает"\ за ростом точного решения. Внося погрешность вида $y(1)=1.0+\Delta y_0$, мы получаем решения, чья первая точка расположена несколько выше, чем первая точка точного решения, тем самым компенсируя более быстрый рост точного решения по сравнению с численным и, при $\Delta y_0 \approx 10^{-1}$ такое возмущение наиболее компенсирует недостаточную скорость роста численного решения, и глобальная погрешность оказывается на 1 порядок меньше, чем при решении задачи Коши с невозмущенным начальным условием. Графики решений и их погрешностей для  начальных условий $y(1)=1.0$ и $y(1)=1.0+0.1$ приведены на рисунке \ref{fig:disturbed}.
	
	\begin{figure}[H]\centering
		\includegraphics[width=.49\textwidth]{ans_Euler_disturbed}
		\includegraphics[width=.49\textwidth]{ans_Adams_disturbed}
	\end{figure}
	
	\begin{figure}[H]\centering
		\includegraphics[width=.49\textwidth]{err_Euler_disturbed}
		\includegraphics[width=.49\textwidth]{err_Adams_disturbed}
		\caption{Решения задачи Коши с начальным условием $y(1)=1.0$ и $y(1)=1.0+0.1$ модифицированным методом Эйлера (слева) и явным методом Адамса 2 порядка (справа).}\label{fig:disturbed}
	\end{figure}

	На рисунке \ref{fig:disturbed} видно, что благодаря возмущению начального условия вида $y(1)=1.0 + 0.1$, решение "поднялось"\ вверх и это обстоятельство компенсирует недостаточную скорость роста численного решения. Так для метода Адамса минимум зависимости погрешности от возмущения начального условия расположен близко к точке $\Delta y_0 = 10^{-1}$ и численное решение задачи с возмущенным начальным условием пересекает точное, а погрешность проходит через нуль. Глобальная погрешность за счет возмущения начального условия уменьшилась в $3/0.5=6$ раз.
	
	Для модифицированного метода Эйлера возмущение $\Delta y_0 = 10^{-1}$ в соответствии с рисунком \ref{fig:err_on_dy} оказывается несколько избыточным и для такого возмущения численное решение целиком лежит выше точного, однако глобальная погрешность по модулю также оказалась меньше, чем глобальная погрешность решения невозмущенной задачи.
	
	\section{Выводы.}
	
	По результатам проведенного исследования можно сформулировать следующие выводы:
	\begin{itemize}
		\item Методы Рунге-Кутты и, в частности, модифицированный метод Эйлера основаны на разложении функции $y(x)$ по формуле Тейлора в точке $x+h$. При построении метода подбирается функция, отличающаяся от суммы членов формулы Тейлора на бесконечно малую порядка, заданного требуемым порядком аппроксимации метода.
		\item Методы Адамса основаны на интерполяции функции $F(x)=f(x,y(x))$ многочленом Лагранжа на равномерной сетке и его интегрировании. При построении метода требуемый порядок аппроксимации достигается за счет выбора степени интерполяционного многочлена Лагранжа, или, что равносильно, шаговости метода Адамса.
		\item Явный метод Адамса 2 порядка требует вычисления разгонной точки. Эту точку следует вычислять методом, порядок аппроксимации которого не меньше 2-го. В данной работе для этого использовался модифицированный метод Эйлера -- один из методов Рунге-Кутты 2 порядка.
		\item Модифицированный метод Эйлера требует на каждой итерации дважды вычислять значение функции $f(x,y)$, в то время как явный метод Адамса 2 порядка -- только единожды (кроме вычисления разгонной точки и первой итерации самого метода Адамса).
		\item Модифицированный метод Эйлера дает более точное решение исследуемой задачи Коши, чем явный метод Адамса 2 порядка, для одного и того же шага.
		\item Оба исследуемых метода устойчивы к возмущениям начального условия ввиду того, что решение задачи Коши для ОДУ 1 порядка, разрешенного относительно производной, непрерывно по начальному условию.
		\item При определенных условиях возмущение начального условия может уменьшить глобальную погрешность решения, полученного обоими исследуемыми методами.
	\end{itemize}
	
\end{document}