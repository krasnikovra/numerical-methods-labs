\documentclass[a4paper, 12pt]{article}
\usepackage[utf8]{inputenc}
\usepackage[russian]{babel}
\usepackage{amsfonts}
\usepackage{amsmath} 
\usepackage{mathtools}
\usepackage{indentfirst} 

\usepackage{geometry} % Меняем поля страницы
\geometry{left=2cm}% левое поле
\geometry{right=1.5cm}% правое поле
\geometry{top=1.5cm}% верхнее поле
\geometry{bottom=3cm}% нижнее поле

\begin{document}
	% Титульный лист
	\begin{titlepage}
		\begin{center}
			Санкт-Петербургский политехнический университет Петра Великого \\ Физико-механический институт \\ Высшая школа прикладной математики и вычислительной физики
		\end{center}
		\vspace{10em}
		\begin{center}
			\Large Отчет по лабораторной работе №3 \\ по дисциплине "Численные методы"
		\end{center}
		\vspace{1em}
		\begin{center}
			\Huge Вычисление определенного интеграла с помощью квадратурных формул Ньютона-Котеса. Формула трех восьмых.
		\end{center}
		\vspace{15em}
		{\Large 
			
			Выполнил: студент гр. 5030102/00003 Красников Р.А.
			\vspace{1em}
			
			Преподаватель: Добрецова С.Б.}
		\vspace{\fill}
		\begin{center}
			Санкт-Петербург \\ 2022
		\end{center}
	\end{titlepage}
	\newpage
	
	\section{Формулировка и формализация задачи.}
	
	\subsection{Формулировка задачи.}
	
	Вычислить определенный интеграл заданной функции на отрезке.
	
	Построить графики зависимости фактической ошибки и числа итераций (число разбиений отрезка) от заданной точности и график зависимости фактической ошибки от длины разбиения.
	
	Исследуя последний график, определить порядок точности применяемой формулы и вычислить константу.
	
	Для заданной функции найти выражение для теоретической ошибки при помощи нужных производных и построить на графике для ошибок линии, соответствующие максимальной и минимальной теоретической ошибке.
	
	\subsection{Постановка задачи.}
	
	Дана функция $f\in C(a,b)$ такая, что $\int\limits_a^bf(x)dx<\infty$, и точность $\epsilon$. 
	
	Найти число $I$ такое, что $|I-\int\limits_a^bf(x)dx|<\epsilon$.
	
	\section{Алгоритм и условия его применимости.}
	
	\subsection{Алгоритм вычисления определенного интеграла по формуле трех восьмых.}
	
	\begin{enumerate}
		\item Ввести функцию $f$, отрезок $[a,b]$, точность $\epsilon$.
		\item Положить $m=1$ -- число отрезков разбиения
		\item Вычислить $I_m$ по формуле
		\begin{equation} \label{method}
			I_m=\frac{3h}{8}\bigg(f(x_0)+3f(x_{n-2})+3f(x_{n-1})+f(x_n)+\sum\limits_{i=1}^{\frac{n}{3}-1}(3f(x_{3i-2})+3f(x_{3i-1})+2f(x_{3i}))\bigg),
		\end{equation}
		где $n=3m$, $h=\dfrac{b-a}{n}$, $x_i=a+ih$.
		\item Вычислить $I_{2m}$ по формуле \eqref{method}.
		\item Если
		\begin{equation}
			|I_m-I_{2m}|<15\epsilon
		\end{equation}
		то закончить вычисления и положить $I=I_{2m}$ -- искомое значение интеграла с точностью $\epsilon$; иначе положить $m=2m$ и вернуться к шагу 4.
	\end{enumerate}
	
	\subsection{Условия применимости метода вычисления определенного интеграла по формуле трех восьмых.}
	
	\begin{enumerate}
		\item Функция $f$ определена и интегрируема на $[a,b]$.
	\end{enumerate}
	
	\section{Анализ задачи.}
	
	Для выполнения работы мне была предложена функция
	\begin{equation} \label{f}
		f(x)=x^5-5,2x^3+5,5x^2-7x-3,5, x\in[-3,0;0,7]
	\end{equation}
	
	Оценим фактическую ошибку пользуясь формулой для теоретической ошибки значения определенного интеграла функции $f$, вычисленного по отрезку $[a,b]$
	\begin{equation}
		\exists\xi\in[a,b]:|I-I_m|=\frac{(b-a)^5}{6480m^4}|f^{(4)}(\xi)|
	\end{equation}
	Заменим $|f^{(4)}(\xi)|$ соответствующими оценками:
	\begin{equation}
		\frac{(b-a)^5}{6480m^4}m_4 \leq |I-I_m| \leq \frac{(b-a)^5}{6480m^4}M_4,
	\end{equation}
	где $m_4=\min\limits_{x\in[a,b]}|f^{(4)}(x)|$, $M_4=\max\limits_{x\in[a,b]}|f^{(4)}(x)|$. Для заданной формулой \eqref{f} функции $f$ имеем
	\begin{equation*} 
		\begin{gathered}
			b-a=3,7,\\
			|f^{(4)}(x)|=120|x|,\\
			m_4=\min\limits_{x\in[-3,0;0,7]}120|x|=0,\\
			M_4=\max\limits_{x\in[-3,0;0,7]}120|x|=360
		\end{gathered}
	\end{equation*}
	и соответствующие оценки фактической ошибки:
	\begin{equation}
		0 \leq |I-I_m| \leq \frac{(3,7)^5}{18m^4}
	\end{equation}
	
	\subsection{Проверка существования и единственности решения.}
	\label{solexistance}
	
	Функция $f$, заданная формулой \eqref{f} -- полином. Любой полином интегрируем по любому отрезку $[a,b]$, поэтому решение существует.
	
	Интеграл функции $f$, интегрируемой на отрезке $[a,b]$, единствен.
	
	\section{Проверка условий применимости метода вычисления определенного интеграла по формуле трех восьмых.}
	
	\begin{enumerate}
		\item Функция $f$, заданная формулой \eqref{f}, определена и интегрируема на отрезке $[-3,0;0,7]$. (см. пункт \ref{solexistance})
	\end{enumerate}
	
	Условия применимости метода вычисления определенного интеграла по формуле трех восьмых выполнены.
	
	\section{Тестовый пример.}
	
	TODO
	
	\section{Подготовка контрольных тестов.}
	
	Для построения графиков 3-х полиномов и функции и графиков фактической ошибки были выбраны $n=3,4,5$ (4,5,6 узлов соответственно).
	
	Для модификации сетки будет исследована зависимость максимальной ошибки в серединах между узлами от параметра $\alpha\in[0,1;3]$ для $n=4$ для следующей сетки:
	\begin{equation} \label{grid}
		x_i=a+(b-a)\bigg(\frac{i}{n}\bigg)^\alpha
	\end{equation}
	
	\section{Модульная структура программы.}
	Для проведения исследования была написана программа на языке C++, состоящая из следующих функций:
	\begin{itemize}
		\item MakeUniformGrid(вх.: отрезок $[a,b]$, натуральное число $n$, параметр $\alpha$; вых.: сетка $\{x_i\}_{i=0}^n$) -- вычисляет сетку $\{x_i\}_{i=0}^n$ по формуле \eqref{grid}.
		\item MakeDataOutOfFunc(вх.: сетка $\{x_i\}_{i=0}^n$, функция $f$; вых.: табличная функция \\ $\{(x_i,y_i)\}_{i=0}^n$) -- вычисляет табличную функцию $\{(x_i,y_i)\}_{i=0}^n$, порожденную функцией $f$ на сетке $\{x_i\}_{i=0}^n$.
		\item MakeTridiagonalMatrixOutOfData(вх.: точка $x$, табличная функция $\{(x_i,y_i)\}_{i=0}^n$; вых.: трехдиагональная матрица\footnote{В программе трехдиагональная матрица представлена тремя одномерными массивами.} $A$) -- вычисляет трехдиагональную матрицу $A$ СЛАУ \eqref{slau}.
		\item MakeRightPartVector(вх.: табличная функция $\{(x_i,y_i)\}_{i=0}^n$; вых.: вектор $b$) -- вычисляет вектор $b$ правой части СЛАУ \eqref{slau}.
		\item ThomasAlghorithm(вх.: трехдиагональная матрица $A$, вектор $b$; вых.: вектор $x$) -- вычисляет решение $x$ СЛАУ $Ax=b$ с трехдиагональной матрицей $A$ методом прогонки.
		\item GetSplineMParams(вх.: табличная функция $\{(x_i,y_i)\}_{i=0}^n$; вых.: числа $\{M_i\}_{i=0}^n$) -- вычисляет числа $\{M_i\}_{i=0}^n$, использующиеся для построения сплайна, решая СЛАУ \eqref{slau} методом прогонки и добавляя $M_0,M_n$, вычисленные по формулам \eqref{M}.
		\item EvaluateSpline(вх.: точка $x$, числа $\{M_i\}_{i=0}^n$, табличная функция $\{(x_i,y_i)\}_{i=0}^n$, вых.: число $S_3^1(x)$) -- вычисляет значение кубического сплайна $S_3^1$ в точке $x$.
	\end{itemize}
	
	\section{Анализ результатов.}
	
	\subsection{Анализ результатов интерполяции функции $f(x)=\ctg x-x, x\in[-3,-1]$ кубическими сплайнами для 4, 5 и 6 узлов на равномерной сетке.} 
	
	На графике для трех построенных кубических сплайнов для 4, 5 и 6 узлов и функции $f$ видно, что все три сплайна, неплохо приближают функцию в правой половине отрезка (на ней $f''$ не велика, т.е функция близка к прямой). В левой половине отрезка ошибка значительно больше чем в правой для всех трех сплайнов (для $n=3$: на $[-3,-2]$ ошибка достигает $1,6$, а на $[-2,-1]$ ошибка не превосходит $0,3$). Это наблюдение в дальнейшем было использовано для подбора параметра $\alpha$ сетки \eqref{grid} для уменьшения ошибки для $n=4$.
	
	\subsection{Анализ зависимости максимальной по серединам между узлами сетки ошибки от числа узлов на равномерной сетке.}
	
	На графике зависимости максимальной по серединам между узлами сетки ошибки от числа узлов на равномерной сетке видим, что ошибка монотонно убывает вплоть до $n=100$, причем с увеличением $n$ ошибка убывает несколько медленнее (функция ошибки от $n$ выпуклая вниз). Это означает, что в отличие от полиномиальной интерполяции, расходящейся при больших $n$, интерполяция кубическими сплайнами обладает свойством монотонного убывания ошибки для всех $n\geq 3$.
	
	\subsection{Анализ результатов модификации сетки для уменьшения максимальной по серединам между узлами ошибки для $n=4$.}
	
	Для выбора параметра $\alpha$ из формулы \eqref{grid} было использовано наблюдение из п. 8.1. Ошибка значительно больше в левой половине отрезка, в связи с чем ожидается, что сдвиг точек сетки влево ($\alpha>1$) уменьшит ошибку. Эта гипотеза подтверждается и непосредственным построением графика зависимости максимальной ошибки по серединам между узлами для $n=4$ от $\alpha$. Таким образом, для проведения исследования на модифицированной сетке была выбрана сетка \eqref{grid} с параметром $\alpha=2$.
	
	На графике для фактической ошибки для сплайнов для $n=3,4,5$ на модифицированной сетке, видим, что для $n=4$ ошибка <<выравнивается>> -- на равномерной сетке ее график выглядит как четыре <<холма>> с максимумами $1,2$, $0,2$, $0,1$ и $0,1$ соответственно, а на модифицированной сетке эти максимумы сблизились -- $0,3$, $0,5$, $0,3$ и $0,5$, за счет чего максимальная ошибка уменьшилась. Однако, для $n=3$ такая модификация ожидаемого результата не дала: ошибка всего лишь <<отразилась>> относительно вертикали и ее максимум не уменьшился.
	
	Интересно отметить, что на графике зависимости максимальной по серединам между узлами ошибки видно, что выбранная модификация сетки уменьшает ошибку не только для $n=4$, но и для всех $n\geq 4$, причем для $n\geq20$ ошибка уменьшилась на 3-4 порядка. Так, в случае полиномиальной интерполяции сгущение точек в левой половине отрезка приводило к более раннему возникновению осцилляций корневого полинома и, как следствие, неэффективности подобной модификации сетки для больших $n$. Интерполяция кубическими сплайнами лишена подобного эффекта, поэтому эффективную модификацию сетки удается подобрать не только для одного-двух малых $n$, а для почти любого числа узлов.
	
	\section{Выводы.}
	
	\begin{itemize}
		\item Интерполяция кубическими сплайнами позволяет с неплохой точностью и достаточно быстро (за счет использования метода прогонки) приближать табличные функции.
		\item Интерполяция кубическим сплайном на равномерной сетке лишена главного недостатка полиномиальной интерполяции -- она не расходится для больших $n$.
		\item Ошибку интерполяции кубическим сплайном можно уменьшить, модифицируя сетку по формуле \eqref{grid} так, чтобы сместить точки сетки в часть отрезка с большей ошибкой.
		\item Для кубических сплайнов существует несколько вариантов граничных условий, один из которых - условие отсутствия узла, состоящее в требовании непрерывности третьей производной во втором и предпоследнем узлах, которое фактически означает полное совпадение первого и второго, предпоследнего и последнего кубических полиномов, составляющих сплайн.
	\end{itemize}
	
\end{document}