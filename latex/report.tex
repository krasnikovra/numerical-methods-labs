\documentclass[a4paper, 12pt]{article}
\usepackage[utf8]{inputenc}
\usepackage[russian]{babel}
\usepackage{amsfonts}
\usepackage{amsmath} 
\usepackage{mathtools}
\usepackage{indentfirst} 

\usepackage{geometry} % Меняем поля страницы
\geometry{left=2cm}% левое поле
\geometry{right=1.5cm}% правое поле
\geometry{top=1.5cm}% верхнее поле
\geometry{bottom=3cm}% нижнее поле

\begin{document}
	% Титульный лист
	\begin{titlepage}
		\begin{center}
			Санкт-Петербургский политехнический университет Петра Великого \\ Физико-механический институт \\ Высшая школа прикладной математики и вычислительной физики
		\end{center}
		\vspace{10em}
		\begin{center}
			\Large Отчет по лабораторной работе №1 \\ по дисциплине "Численные методы"
		\end{center}
		\vspace{1em}
		\begin{center}
			\Huge Интерполяционные полиномы приближения табличных функций. Полином в форме Эрмита.
		\end{center}
		\vspace{15em}
		{\Large 
			
			Выполнил: студент гр. 5030102/00003 Красников Р.А.
			\vspace{1em}
			
			Преподаватель: Добрецова С.Б.}
		\vspace{\fill}
		\begin{center}
			Санкт-Петербург \\ 2022
		\end{center}
	\end{titlepage}
	\newpage
	
	\section{\underline{Формулировка и формализация задачи.}}
	
	\subsection{Формулировка задачи.}
	
	Интерполировать табличную функцию, порожденную заданной гладкой функцией, полиномом в форме Эрмита.
	
	Построить графики нескольких полиномов для равномерных сеток с различным числом узлов. Исследовать для этих полиномов функцию фактической ошибки. Исследовать зависимость максимальной ошибки на отрезке в зависимости от числа узлов.
	
	\subsection{Постановка задачи.}
	
	Дана гладкая функция $f:[a,b]\rightarrow \mathbb{R}$ и совокупность троек значений $\{(x_i,y_i,y'_i)\}_{i=0}^n$ такая, что $y_i=f(x_i), y'_i=f'(x_i)$ и все $x_i$ попарно различны. 
	
	Найти полином $P_{2n+1}(x)$ такой, что $\forall i=\overline{0,n} : P_{2n+1}(x_i)=y_i, P'_{2n+1}(x_i) =y'_i$.
	
	\section{\underline{Алгоритм и условия его применимости.}}
	
	\subsection{Алгоритм интерполяции гладкой функции полиномом Эрмита на равномерной сетке.}
	
	\begin{enumerate}
		\item Ввести функцию $f$ и ее производную $f'$, отрезок $[a,b]$, число $n$ (на единицу меньшее числа узлов).
		\item Вычислить узлы равномерной сетки по формуле
		\begin{equation}
			x_i=a+\frac{b-a}{n}i, i=\overline{0,n}
		\end{equation}
		\item Вычислить значения функции $f$ и ее производной $f'$ в узлах сетки по формулам
		\begin{equation}
			\begin{aligned}
			&y_i=f(x_i),\\
			&y'_i=f'(x_i), i=\overline{0,n}
			\end{aligned}
		\end{equation}
		\item Построить\footnote{Построить в описанном алгоритме означает не вычислить коэффициенты полинома в каноническом виде, а использовать данную формулу для вычисления значения полинома в конкретной точке $x$.} интерполяционный полином $P_{2n+1}(x)$ по формуле
		\begin{equation}
			P_{2n+1}(x)=\sum\limits_{j=0}^n\bigg[(x-x_j)y'_j+\bigg(1-2\sum\limits_{k=0,k\neq j}^n\frac{x-x_j}{x_j-x_k}\bigg)y_j\bigg]\prod\limits_{i=0,i\neq j}^n\bigg(\frac{x-x_i}{x_j-x_i}\bigg)^2
		\end{equation}	
		
	\end{enumerate}

	\subsection{Условия применимости метода интерполяции полиномом Эрмита.}
	
	\begin{enumerate}
		\item Функция $f$ дифференцируема\footnote{Это условие требуется для возможности вычислить все $y'_i$.} на $[a,b]$ (дифференцируемость в концевых точках понимается в одностороннем смысле).
		\item Узлы сетки попарно различны (для описанного выше алгоритма с использованием равномерной сетки выполняется).
	\end{enumerate}

	\section{\underline{Анализ задачи.}}
	
	Для выполнения работы мне была предложена функция
	\begin{equation}
		\begin{aligned}
				&f(x)=\ctg x - x\\
			&f'(x)=-\frac{1}{\sin^2x}-1
		\end{aligned}
	\end{equation}
	Для ее интерполяции был выбран отрезок $[-3,-1]$.
	
	\subsection{Проверка существования и единственности решения.}
	
	Точки разрыва функции $f$: $\overline{x}_k=\pi k,k\in\mathbb{Z}$. На интервалах $(\overline{x}_{i}, \overline{x}_{i+1})$ функция $f$ гладкая. Для соседних точек разрыва $\overline{x}_{-1}=-\pi$ и $\overline{x}_{0}=0$ справедливо включение $[-3,-1]\subset(-\pi,0)$, поэтому функция $f$ гладкая на отрезке $[-3,-1]$.
	
	Поскольку для любой сетки $\{x_i\}_{i=0}^n$, в которой все $x_i$ попарно различны и $x_i\in[-3,-1]$ существуют $y_i=f(x_i), y'_i=f'(x_i)$, то существует и единствен полином $P_{2n+1}(x)$ такой, что $\forall i=\overline{0,n} : P_{2n+1}(x_i)=y_i, P'_{2n+1}(x_i) =y'_i$.
	
	\section{\underline{Проверка условий применимости метода интерполяции полиномом Эрмита.}}
	
	\begin{enumerate}
		\item Функция $f(x)=\ctg x - x$ гладкая на отрезке $[-3,-1]$. (см. предыдущий пункт)
		\item Исследование проводится для равномерных сеток, поэтому все узлы сетки попарно различны
	\end{enumerate}
	
	Все условия применимости метода интерполяции полиномом Эрмита выполнены.
	
	\section{\underline{Тестовый пример.}}
	
	Для демонстрации метода интерполяции полиномом Эрмита интерполируем функцию $f(x)=\sqrt{x}$  $(f'(x)=\dfrac{1}{2\sqrt{x}})$ на отрезке $[1,9]$ на равномерной сетке для $n=2$:
	\begin{center}
		\begin{tabular}{ | c | c | c | c | }
			\hline
			$i$ & $x_i$ & $y_i$ & $y'_i$ \\ \hline
			0 & 1 & 1 & 0,5 \\ \hline
			1 & 5 & 2,2361 & 0,2236 \\ \hline
			2 & 9 & 3 & 0,1667 \\
			\hline
		\end{tabular}
	\end{center}
	Обозначим 
	\begin{equation}
		a_j(x)=\bigg[(x-x_j)y'_j+\bigg(1-2\sum\limits_{k=0,k\neq j}^2\frac{x-x_j}{x_j-x_k}\bigg)y_j\bigg]\prod\limits_{i=0,i\neq j}^2\bigg(\frac{x-x_i}{x_j-x_i}\bigg)^2
	\end{equation}
	Тогда искомый полином $P_5(x)$ запишется в виде:
	\begin{equation}
		P_5(x)=a_0(x)+a_1(x)+a_2(x)
	\end{equation}

	Вычислим поочередно все $a_j(x)$:
	\begin{itemize}
		\item $j=0$:
		\begin{equation*}
			\begin{aligned}
				a_0(x)&=\bigg[(x-x_0)y'_0+\bigg(1-2\sum\limits_{k=1}^2\frac{x-x_0}{x_0-x_k}\bigg)y_0\bigg]\prod\limits_{i=1}^2\bigg(\frac{x-x_i}{x_0-x_i}\bigg)^2\\&=\bigg[(x-1)\cdot0,5+\bigg(1-2\bigg(\frac{x-1}{1-5}+\frac{x-1}{1-9}\bigg)\bigg)\cdot1\bigg]\bigg(\frac{x-5}{1-5}\bigg)^2\bigg(\frac{x-9}{1-9}\bigg)^2\\
				&=0,001220703125x^5\\
				&-0,034423828125x^4\\
				&+0,355957031250x^3\\
				&-1,607910156250x^2\\ 
				&+2,779541015620x\\
				&-0,494384765625
			\end{aligned}
		\end{equation*}
		\item $j=1$:
		\begin{equation*}
			\begin{aligned}
				a_1(x)&=\bigg[(x-x_1)y'_1+\bigg(1-2\sum\limits_{k=0,k\neq 1}^2\frac{x-x_1}{x_1-x_k}\bigg)y_1\bigg]\prod\limits_{i=0,i\neq 1}^2\bigg(\frac{x-x_i}{x_1-x_i}\bigg)^2\\&=\bigg[(x-5)\cdot0,2236+\bigg(1-2\bigg(\frac{x-5}{5-1}+\frac{x-5}{5-9}\bigg)\bigg)\cdot2,2361\bigg]\bigg(\frac{x-1}{5-1}\bigg)^2\bigg(\frac{x-9}{5-9}\bigg)^2\\
				&=0,000873437500x^5\\
				&-0,013101171875x^4\\
				&+0,015714062500x^3\\
				&+0,358155468750x^2\\ 
				&-0,715415625000x\\
				&+0,353773828125
			\end{aligned}
		\end{equation*}
		\item $j=2$:
		\begin{equation*}
			\begin{aligned}
				a_2(x)&=\bigg[(x-x_2)y'_2+\bigg(1-2\sum\limits_{k=0}^1\frac{x-x_2}{x_2-x_k}\bigg)y_2\bigg]\prod\limits_{i=0}^1\bigg(\frac{x-x_2}{x_2-x_i}\bigg)^2\\&=\bigg[(x-9)\cdot0,1667+\bigg(1-2\bigg(\frac{x-9}{9-1}+\frac{x-9}{9-5}\bigg)\bigg)\cdot3\bigg]\bigg(\frac{x-1}{9-1}\bigg)^2\bigg(\frac{x-5}{9-5}\bigg)^2\\
			   =&-0,002034472656x^5\\
				&+0,045653613281x^4\\
				&-0,348465039062x^3\\
				&+1,099105664060x^2\\ 
				&-1,325258300780x\\
				&+0,530998535156
			\end{aligned}
		\end{equation*}
		Складывая все $a_j(x)$ получаем искомый полином:
		\begin{equation*}
			\begin{aligned}
				P_5(x)&=0,000059667970x^{5}\\
					  &-0,001871386719x^{4}\\
					  &+0,023206054688x^{3}\\
					  &-0,150649023440x^{2}\\
					  &+0,738867089840x\\
					  &+0,390387597656
			\end{aligned}
		\end{equation*}
	\end{itemize}
	
	Ошибки в узлах:
	\begin{equation*}
		\begin{aligned}
			&|P_5(1)-f(1)|=5,0\cdot10^{-12}\\
			&|P_5(5)-f(5)|=3,2\cdot10^{-5}\\
			&|P_5(9)-f(9)|=7,2\cdot10^{-8}
		\end{aligned}
	\end{equation*}

	Ошибки в серединах между узлами:
	\begin{equation*}
		\begin{aligned}
			&|P_5(3)-f(3)|=8,6\cdot10^{-3}\\
			&|P_5(7)-f(7)|=4,2\cdot10^{-3}
		\end{aligned}
	\end{equation*}
	
	\section{\underline{Подготовка контрольных тестов.}}
	
    Для построения графиков 3-х полиномов и функции и графиков фактической ошибки были выбраны $n=2,3,4$ (3,4,5 узлов соответственно). Фактическая ошибка сравнивается с теоретической, заданной формулой
    \begin{equation} \label{theor_err}
    	\exists \eta \in[a,b]:R_{2n+1}(x)=|P_{2n+1}(x)-f(x)|=\frac{f^{(2n+2)}(\eta)}{(2n+2)!}\omega^2_{n+1}(x),
    \end{equation} 
	где $\displaystyle \omega_{n+1}(x)=\prod\limits_{i=0}^n(x-x_i)$ -- корневой полином.
	
	Для модификации сетки будет исследована зависимость максимальной ошибки в серединах между узлами от параметра $\alpha\in[0,1;3]$ для $n=2$ для следующей сетки:
    \begin{equation} \label{grid}
    	x_i=a+(b-a)\bigg(\frac{i}{n}\bigg)^\alpha
    \end{equation}
	
	\section{\underline{Модульная структура программы.}}
	Для проведения исследования была написана программа на языке C++, состоящая из следующих функций:
	\begin{itemize}
		\item MakeUniformGrid(вх.: отрезок $[a,b]$, натуральное число $n$, параметр $\alpha$; вых.: сетка $\{x_i\}_{i=0}^n$) -- вычисляет сетку $\{x_i\}_{i=0}^n$ по формуле \eqref{grid}.
		\item MakeDataOutOfFunc(вх.: сетка $\{x_i\}_{i=0}^n$, функции $f$ и $f'$; вых.: табличная функция \\ $\{(x_i,y_i,y'_i)\}_{i=0}^n$) -- вычисляет табличную функцию $\{(x_i,y_i,y'_i)\}_{i=0}^n$, порожденную функцией $f$ и ее производной $f'$ на сетке $\{x_i\}_{i=0}^n$.
		\item EvaluateHermitePolynomial(вх.: точка $x$, табличная функция $\{(x_i,y_i,y'_i)\}_{i=0}^n$; вых.: число $P_{2n+1}(x)$) -- вычисляет значение интерполяционного полинома Эрмита для табличной функции $\{(x_i,y_i,y'_i)\}_{i=0}^n$ в точке $x$.
		\item EvaluateMaxMidpointsError(вх.: табличная функция $\{(x_i,y_i,y'_i)\}_{i=0}^n$, функция $f$; вых.: число $\delta_{2n+1}$) -- вычисляет максимальную по серединам между узлами сетки фактическую ошибку приближения функции $f$ полиномом Эрмита для табличной функции $\{(x_i,y_i,y'_i)\}_{i=0}^n$.
		\item EvaluateTheoreticError(вх.: точки $x$ и $\eta$, сетка $\{x_i\}_{i=0}^n$, функция $f^{(2n+2)}$; вых.: число $R_{2n+1}(x)$) -- вычисляет значение функции $R_{2n+1}$ в точке $x$ по формуле \eqref{theor_err}.
	\end{itemize}
	
	\section{\underline{Анализ результатов.}}
	
	\subsection{Анализ результатов интерполяции функции $f(x)=\ctg x-x, x\in[-3,-1]$ полиномами Эрмита для 3, 4 и 5 узлов на равномерной сетке.} 
	
	На графике для трех построенных полиномов Эрмита для 3, 4 и 5 узлов и функции $f$ видно, что полиномы, построенные для 3 и 4 узлов интерполяции не очень хорошо приближают функцию $f$ (ошибка достигает приблизительно $2,4$ и $0,8$ для 3 и 4 узлов соответственно), хотя и проходят через узлы интерполяции под правильным наклоном. Полином же для 5 узлов уже более точно приближает фукнцию (ошибка не превосходит $0,3$), о чем свидетельствует как график для полиномов и функции $f$, так и график фактической ошибки для всех трех полиномов. В левой половине отрезка ошибка больше чем в правой для всех трех полиномов. Стоит отметить, что функция ошибки проходит через локальные минимумы в узлах гладко (<<без угла>>), что достигается благодаря тому, что интерполяция полиномом Эрмита учитывает значения производной функции $f$ в точках сетки. Линия для теоретической ошибки для $n=2$ вполне похожа на линию фактической ошибки для того же $n$.
	
	\subsection{Анализ зависимости максимальной по серединам между узлами сетки ошибки от числа узлов на равномерной сетке.}
	
	На графике зависимости максимальной по серединам между узлами сетки ошибки от числа узлов на равномерной сетке видим, что ошибка монотонно убывает вплоть до $n=22$, однако для $n>22$ она начинает возрастать и в точке $n=37$ становится уже на порядок больше, чем в точке $n=1$. Это объясняется тем, что корневой полином, построенный для равномерной сетки с ростом $n$ начинает очень сильно отклоняться от 0, поэтому, согласно формуле \eqref{theor_err}, может случиться так, что из-за данного свойства корневого полинома ошибка и вовсе начнет возрастать с некоторого числа $n$, что и наблюдается на графике.
	
	\subsection{Анализ результатов модификации сетки для уменьшения максимальной по серединам между узлами ошибки для $n=2$.}
	
	Для выбора параметра $\alpha$ из формулы \eqref{grid} был построен график зависимости максимальной по серединам между узлами ошибки от $\alpha$ для $n=2$. На нем видно, что при $\alpha=1,23$ ошибка меньше, чем для равномерной сетки ($\alpha=1$). Поэтому был выбран параметр $\alpha=1,23$.
	
	На графике для фактической ошибки для полиномов для $n=2,3,4$ на модифицированной сетке, видим, что для $n=2$ ошибка <<выравнивается>> -- на равномерной сетке ее график выглядит как два <<холма>> с максимумами $2,4$ и $0,7$ соответственно, а на модифицированной сетке эти максимумы сблизились -- $1,9$ и $1,6$, за счет чего максимальная ошибка уменьшилась.
	
	Однако, на графике зависимости максимальной по серединам между узлами сетки ошибки от числа узлов на модифицированной сетке видим, что модифицированная сетка почти везде проигрывает равномерной, причем на 2-3 порядка. Построив график зависимости максимальной по серединам между узлами ошибки от $\alpha$ для $n=22$, можно убедиться, что для больших $n$ оптимальным параметром все же является $\alpha=1$, тем не менее для малых $n$ оказывается возможным сместить точки в части отрезка с большей ошибкой и тем самым уменьшить максимальную по серединам между узлами ошибку.
	
	\section{\underline{Выводы.}}
	
	\begin{itemize}
		\item Полиномы Эрмита позволяют с неплохой точностью приближать функции, для которых можно вычислить производную, даже для небольшого числа узлов. Однако, недостаток этих полиномов состоит в том, что не для всех функций можно легко найти аналитическое выражение для производной.
		\item Интерполяция на равномерной сетке имеет предел точности, достигаемый при некотором $n$, после которого увеличение числа узлов ведет лишь к увеличению ошибки. Это следствие формулы \eqref{theor_err} и свойств корневого полинома.
		\item Для небольших $n$ можно уменьшить максимальную ошибку интерполяции, подобрав параметр $\alpha$ для модификации равномерной сетки \eqref{grid}.
	\end{itemize}
	
\end{document}