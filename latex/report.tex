\documentclass[a4paper, 12pt]{article}
\usepackage[utf8]{inputenc}
\usepackage[russian]{babel}
\usepackage{amsfonts}
\usepackage{amsmath} 
\usepackage{amssymb}
\usepackage{mathtools}
\usepackage{indentfirst} 

\usepackage{geometry} % Меняем поля страницы
\geometry{left=2cm}% левое поле
\geometry{right=1.5cm}% правое поле
\geometry{top=1.5cm}% верхнее поле
\geometry{bottom=3cm}% нижнее поле

\newtheorem{theorem}{Теорема}

\begin{document}
	% Титульный лист
	\begin{titlepage}
		\begin{center}
			Санкт-Петербургский политехнический университет Петра Великого \\ Физико-механический институт \\ Высшая школа прикладной математики и вычислительной физики
		\end{center}
		\vspace{10em}
		\begin{center}
			\Large Отчет по лабораторной работе №6 \\ по дисциплине "Численные методы"
		\end{center}
		\vspace{1em}
		\begin{center}
			\Huge Решение задачи Коши для ОДУ 1 порядка. Явный метод Адамса 2 порядка.
		\end{center}
		\vspace{15em}
		{\Large 
			
			Выполнил: студент гр. 5030102/00003 Красников Р.А.
			\vspace{1em}
			
			Преподаватель: Добрецова С.Б.}
		\vspace{\fill}
		\begin{center}
			Санкт-Петербург \\ 2022
		\end{center}
	\end{titlepage}
	\newpage
	
	\section{Формулировка и формализация задачи.}
	
	\subsection{Формулировка задачи.}
	
	Решить задачу Коши для ОДУ 1 порядка, разрешенного относительно производной.
	
	Построить графики точного и полученного решения, а также график ошибки для двух значений шага.
	
	Построить график зависимости фактической точности от величины шага. По графику определить порядок точности метода и константу.
	
	Построить аналогичные графики для схемы предиктор-корректор 2 порядка и сравнить ее с явным методом Адамса 2 порядка.
	
	\subsection{Постановка задачи.}
	
	Дана задача Коши
	\begin{equation} \label{cauchytask}
		\begin{gathered}
			y'=f(x,y), x\in[a,b],\\
			y(a)=y_0
		\end{gathered}
	\end{equation}
	Ее решение -- непрерывно дифференцируемая функция $\varphi$ такая, что
	\begin{equation}
		\begin{gathered}
			\forall x\in[a,b]: \varphi'(x)=f(x,\varphi(x)),\\ 
			\varphi(a)=y_0
		\end{gathered}
	\end{equation}
	Также задана сетка $\{x_i\}_{i=0}^n$.
	
	Найти сеточную функцию $\{(x_i,y_i)\}_{i=0}^n$ такую, что
	\begin{equation}
		y_i\approx\varphi(x_i), i=\overline{0,n}
	\end{equation}
	
	\section{Алгоритм и условия его применимости.}
	
	\subsection{Алгоритм решения задачи Коши \eqref{cauchytask} явным методом Адамса 2 порядка.}
	\label{alghorithm_simple}
	
	\begin{enumerate}
		\item Ввести функцию $f(x,y)$, начальное условие $y_0$, отрезок $[a,b]$, число отрезков разбиения $m$ (число точек равномерной сетки составит $m+1$).
		\item Вычислить разгонную точку (с помощью модифицированного метода Эйлера из лабораторной работы №5):
		\begin{equation} \label{euler}
			y_1 = y_0 + hf(x_0 + \frac{h}{2}, y_0+\frac{h}{2}f(x_0,y_0)),
		\end{equation}
		где $h=\dfrac{b-a}{m}$, $x_i=a+ih$.
		\item Для всех $i=\overline{2,m}$ вычислить
		\begin{equation} \label{method}
			y_i = y_{i-1} + \frac{h}{2}(3f(x_{i-1},y_{i-1})-f(x_{i-2}, y_{i-2}))
		\end{equation}
		\item $\{(x_i,y_i)\}_{i=0}^m$ -- искомая сеточная функция
	\end{enumerate}
	
	\textit{Замечание.} Можно задавать не число отрезков разбиения (или число точек равномерной сетки), а непосредственно сам шаг $h$, но в этом случае не гарантируется, что полученная сеточная функция будет определена в точке $b$, то есть формально решение будет найдено для некоторого меньшего отрезка ${[a,x_m] \varsubsetneq [a,b]}$.
	
	\subsection{Построение формулы \eqref{method}.}
	
	Метод Адамса -- частный случай конечно-разностных методов:
	\begin{equation}
		\sum\limits_{j=0}^ra_jy_{k-j}=h\sum\limits_{j=0}^rb_jf(x_{k-j})
	\end{equation}

	Явный метод Адамса -- это конечно-разностный метод с $a_0=1,a_1=-1,a_j=0,j\geq2$ и $b_0=0$.
	
	Для нахождения $a_j,b_j$ необходимо решить СЛАУ
	\begin{equation} \label{slau}
		\begin{cases}
			\sum\limits_{j=0}^ra_j=0,\\
			\sum\limits_{j=0}^rja_j=-1,\\
			\sum\limits_{j=0}^rb_j=1,\\
			\sum\limits_{j=0}^r(j^ia_j+ij^{i-1}b_j)=0, i=2,...,m
		\end{cases}
	\end{equation}
	где $m$ -- порядок метода.
	
	Для явного метода Адамса 2 порядка СЛАУ \eqref{slau} принимает вид
	\begin{equation}
		\begin{cases}
		a_0+a_1+a_2=0,\\
		a_1+2a_2=-1,\\
		b_0+b_1+b_2=0,\\
		a_1+2b_1+4a_2+4b_2=0
		\end{cases}
	\end{equation}
	
	Первые 2 уравнения в соответствии с определением явного метода Адамса -- это тождества. Остаются 2 уравнения для 2 неизвестных $b_1,b_2$:
	\begin{equation}
		\begin{cases}
		b_1+b_2=0,\\
		2b_1+4b_2=1
		\end{cases}
		\Leftrightarrow
		\begin{cases}
			b_1=\dfrac{3}{2},\\
			b_2=-\dfrac{1}{2}
		\end{cases}
	\end{equation}

	Получили формулу
	\begin{equation}
		y_k=y_{k-1}+h(\frac{3}{2}f(x_{k-1},y_{k-1})-\frac{1}{2}f(x_{k-2},y_{k-2}))
	\end{equation}
	
	\subsection{Условия применимости алгоритма \ref{alghorithm_simple} решения задачи Коши \eqref{cauchytask} явным методом Адамса 2 порядка.}
	
	\begin{enumerate}
		\item Задача Коши \eqref{cauchytask} имеет единственное решение $y=\varphi(x), x\in[a,b]$.
	\end{enumerate}
	
	В отчете к лабораторной работе №5 было показано, что в качестве достаточного условия применимости метода удобно использовать следующее условие:
	\begin{equation*}
		\forall [c,d]\subset\mathbb{R}, f(x,y),\dfrac{\partial f}{\partial y}(x,y) \text{ определены на } \Pi=[a,b]\times[c,d] : f(x,y), \dfrac{\partial f}{\partial y}(x,y) \text{ непрерывны в } \Pi 
	\end{equation*}
	
	\section{Анализ задачи.}
	
	Для выполнения работы мне была предложена задача Коши
	\begin{equation} \label{mytaskbad}
		\begin{gathered}
			xy'-2x^2\sqrt{y}=4y, x\in[1,2],\\
			y(1)=0
		\end{gathered}
	\end{equation}
	и ее точное решение $y=x^4\ln^2x$. Однако нетрудно видеть, что $y=0$ также является решением задачи Коши \eqref{mytaskbad} и $\exists x\in[1,2]: x^4\ln^2x\neq0$, то есть решение задачи Коши \eqref{mytaskbad} не единственно.
	
	Поэтому исследование выполнено для задачи Коши
	\begin{equation} \label{mytaskinit}
		\begin{gathered}
			xy'-2x^2\sqrt{y}=4y, x\in[1,2],\\
			y(1)=1
		\end{gathered}
	\end{equation}
	и ее точного решения $y^*=x^4(\ln x+1)^2$. 
	
	\subsection{Проверка существования и единственности решения.}
	\label{solexistance}
	
	Приведем задачу Коши \eqref{mytaskinit} к виду \eqref{cauchytask}
	\begin{equation} \label{mytask}
		\begin{gathered}
			y'=2x\sqrt{y}+4\frac{y}{x}, x\in[1,2],\\
			y(1)=1
		\end{gathered}
	\end{equation}
	Ее точное решение $y^*=x^4(\ln x+1)^2$. В отчете к лабораторной работе №5 была доказана единственность решения задачи Коши \eqref{mytask}.
	
	\section{Проверка условий применимости явного метода Адамса 2 порядка.}
	
	\begin{enumerate}
		\item Решение задачи Коши \eqref{mytask} единственно. (см. пункт \ref{solexistance}).
	\end{enumerate}
	
	Условия применимости модифицированного метода Эйлера выполнены.
	
	\section{Тестовый пример.} \label{testexample}
	
	Для демонстрации явного метода Адамса вычислим приближенное значение решения задачи Коши \eqref{mytask} в точке $1.2$ с шагом $0.1$. Такие значения выбраны потому, что точка $1.1$ окажется лишь разгонной и значение сеточной функции в ней будет вычислено по модифицированному методу Эйлера (вариант лабораторной работы №5).
	
	Имеем, для начала:
	\begin{equation*}
		\begin{gathered}
			f(x,y) = 2x\sqrt{y}+4\frac{y}{x},\\
			x_0 = a = 1,\\
			y_0 = y(a) = 1
		\end{gathered}
	\end{equation*}
	
	\begin{enumerate}
		\item Вычислим разгонную точку:
		\begin{equation*}
			\begin{aligned}
				x_1 &= x_0 + h = 1 + 0.1 = 1.1,\\
				y_1 &= y_0 + hf(x_0 + \frac{h}{2}, y_0+\frac{h}{2}f(x_0,y_0)) \\&= 1 + 0.1f(1 + 0.05, 1 + 0.05f(1,1)) = 1.7347
			\end{aligned}
		\end{equation*}
		\item Вычислим следующую точку по формуле \eqref{method} явного метода Адамса 2 порядка:
		\begin{equation*}
			\begin{aligned}
				x_2 &= x_0 + 2h = 1 + 2\cdot0.1 = 1.2,\\
				y_2 &= y_1 + \frac{h}{2}(3f(x_1,y_1)-f(x_0,y_0)) \\&= 1.7347 + 0.05(3f(1.1,1.7347)-f(1,1)) = 2.8155
			\end{aligned}
		\end{equation*}
	\end{enumerate} 
	
	Точное решение $y^*=x^4(\ln x+1)^2$. Фактическая ошибка:
	\begin{equation*}
		\begin{gathered}
			|y^*(x_2)-y_2|=0.0832
		\end{gathered}
	\end{equation*}

	В лабораторной работе №5 при вычислении $y_2$ с шагом $0.1$ модифицированным методом Эйлера ошибка составила $0.0598$.
	
	\section{Подготовка контрольных тестов.}
	
	Для построения графиков полученных решений и ошибок были выбраны значения шага $h_1=0.2$ и $h_2=0.1$
	
	Для построения графика зависимости фактической точности от величины шага были выбраны значения шага $h_i=2^{-i}, i=\overline{0,14}$
	
	Для сравнения с явным методом была реализована также схема предиктор-корректор: предиктор -- модифицированный метод Эйлера, корректор -- неявная формула Адамса 2 порядка. Алгоритм вычисления похож на \ref{alghorithm_simple} с той разницей, что для всех $i=\overline{1,m}$ необходимо вычислять $y_i$ по формулам
	\begin{equation} \label{predcorr}
		\begin{aligned}
			&\widetilde{y}_i = y_{i-1} + hf(x_{i-1} + \frac{h}{2}, y_{i-1}+\frac{h}{2}f(x_{i-1},y_{i-1})),\\
			&y_i = y_{i-1} + \frac{h}{2}(f(x_i,\widetilde{y}_i) + f(x_{i-1}, y_{i-1}))
		\end{aligned}
	\end{equation}
	
	\section{Модульная структура программы.}
	Для проведения исследования была написана программа на языке C++, состоящая из следующих функций:
	\begin{itemize}
		\item EulerNextY(вх.: функция $f(x,y)$, числа $x_{i-1},y_{i-1}$, шаг $h$; вых.: число $y_i$) -- вычисляет значение $y_i$ по формуле \eqref{euler} (нужна для избежания дублирования кода в следующих функциях).
		\item AdamsODE(вх.: функция $f(x,y)$, нач. условие $y_0$, отрезок $[a,b]$, число отрезков разбиения $n$; вых.: сеточная функция $\{(x_i,y_i)\}_{i=0}^n$) -- вычисляет численное решение задачи Коши $y'=f(x,y),x\in[a,b]$ с начальным условием $y(a)=y_0$ по алгоритму \ref{alghorithm_simple}.
		\item EulerAdamsPredictorCorrectorODE(вх.: функция $f(x,y)$, нач. условие $y_0$, отрезок $[a,b]$, число отрезков разбиения $n$; вых.: сеточная функция $\{(x_i,y_i)\}_{i=0}^n$) -- вычисляет численное решение задачи Коши $y'=f(x,y),x\in[a,b]$ с начальным условием $y(a)=y_0$ по формулам \eqref{predcorr}.
	\end{itemize}
	
	\section{Анализ результатов.}
	
	\subsection{Анализ графиков полученного решения.} \label{ans_analytics}
	
	На графике полученного и точного решений видим, что функция $y^*$ являющаяся точным решением возрастает на $[1,2]$ и выпукла на $[1,2]$. Точки сеточной функции представляют похожую зависимость, но с ростом $x$ начинают отдаляться от точного решения в меньшую сторону, как бы не успевая за ростом функции $y^*$. Решение, полученное для шага $h_2=0.1$ ближе к точному, чем полученное для шага $h_1=0.2$.
	
	На графике ошибок полученных решений видим, что с ростом $x$ ошибка возрастает, причем не по линейному закону, а более быстро. Для шага $h_2=0.1$ ошибка меньше и возрастает медленнее, чем для шага $h_1=0.2$. 
	
	В точке $x=2$ ошибка составила $11.16$ и $3.63$ для шагов $h_1=0.2$ и $h_2=0.1$ соответственно. При уменьшении шага в 2 раза ошибка уменьшилась в $3.07$ раз, несколько меньше ожидаемого (4 раза). Это объясняется тем, что ошибка выражается как $Ch^2$ только при достаточно малых $h$, а $h_1,h_2$ еще достаточно велики.
	
	\subsection{Анализ зависимости фактической ошибки от величины шага.}
	
	На графике зависимости фактической ошибки от величины шага видим, что линия ошибки представляет из себя прямую для $h\leq10^{-2}$, а для больших $h$ отклоняется вогнуто от прямой. Это подтверждает гипотезу высказанную в пункте \ref{ans_analytics} о причинах того, что ошибка уменьшилась не достаточно сильно при уменьшении шага в 2 раза. По графику были определены порядок метода -- 2 и константа -- $410.34$, т.е.
	\begin{equation}
		\delta_{\text{max}}\approx410.34h^2
	\end{equation}
	Для сравнения, в лабораторной работе №5 для модифицированного метода Эйлера константа составила $228.98$. Явный метод Адамса 2 порядка имеет тот же порядок, что и модифицированный метод Эйлера, но большую константу. Если принять во внимание то, что в отличие от модифицированного метода Эйлера явный метод Адамса 2 порядка требует только 1 (вместо 2 в методе Эйлера) вычисления значения функции $f(x,y)$ на каждой (кроме первой) итерации, то такой результат выглядит вполне логично -- объем вычислений меньше, но сходимость несколько хуже.

	\subsection{Сравнение с результатами исследования для схемы предиктор-корректор.}
	
	На графике полученного (схемой предиктор-корректор) и точного решений видим, что данная схема дала для исследуемой задачи более точное решение при тех же значениях шага (для шага $h1=0.2$ в точке $x=2$ схема предиктор-корректор дала ошибку $0.4$, а явный метод Адамса 2 порядка -- $11.2$). Это ожидаемый результат, поскольку явный метод Адамса по показателям сходимости оказался близок к модифицированному методу Эйлера, использованному лишь в качестве предиктора.
	
	На графике зависимости фактической ошибки (схемы предиктор-корректор) от величины шага видим, что линия ошибки аналогично и почти в том же месте (при $h=3\cdot10^{-3}$) начинает отклоняться от прямой, но в точке $h=0.25$ имеет локальный минимум. Константа для схемы предиктор-корректор составила $81.97$.
	
	\section{Выводы.}
	
	\begin{itemize}
		\item Явные методы Адамса требуют вычисления разгонных точек с помощью других методов (например, методов Рунге-Кутты).
		\item Явный метод Адамса 2 порядка позволяет добиться того же порядка точности, что и модифицированный метод Эйлера, но при этом требует вдвое меньше вычислений значения функции $f(x,y)$.
		\item По скорости сходимости явный метод Адамса 2 порядка несколько уступает модифицированному методу Эйлера (как минимум, в исследованной задаче).
		\item Схема предиктор-корректор 2 порядка сходится быстрее явного метода Адамса, но требует большего числа вычислений значения функции $f(x,y)$.
	\end{itemize}
	
\end{document}