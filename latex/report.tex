\documentclass[a4paper, 12pt]{article}
\usepackage[utf8]{inputenc}
\usepackage[russian]{babel}
\usepackage{amsfonts}
\usepackage{amsmath} 
\usepackage{mathtools}
\usepackage{indentfirst} 

\usepackage{geometry} % Меняем поля страницы
\geometry{left=2cm}% левое поле
\geometry{right=1.5cm}% правое поле
\geometry{top=1.5cm}% верхнее поле
\geometry{bottom=3cm}% нижнее поле

\begin{document}
	% Титульный лист
	\begin{titlepage}
		\begin{center}
			Санкт-Петербургский политехнический университет Петра Великого \\ Физико-механический институт \\ Высшая школа прикладной математики и вычислительной физики
		\end{center}
		\vspace{10em}
		\begin{center}
			\Large Отчет по лабораторной работе №2 \\ по дисциплине "Численные методы"
		\end{center}
		\vspace{1em}
		\begin{center}
			\Huge Решение системы линейных алгебраических уравнений методом LU-разложения
		\end{center}
		\vspace{15em}
		{\Large 
			
			Выполнил: студент гр. 5030102/00003 Красников Р.А.
			\vspace{1em}
			
			Преподаватель: Добрецова С.Б.}
		\vspace{\fill}
		\begin{center}
			Санкт-Петербург \\ 2021
		\end{center}
	\end{titlepage}
	\newpage
	
	\section{\underline{Формулировка и формализация задачи.}}
	
	\subsection{Формулировка задачи.}
	
	Решить систему линейных алгебраических уравнений (далее -- СЛАУ) методом LU-разложения.
	
	Вычислить норму фактической ошибки решения и норму невязки, проверить справедливость оценки относительной погрешности ошибки.
	
	Исследовать зависимость нормы фактической ошибки решения СЛАУ с матрицей Гильберта от размерности матрицы Гильберта и зависимость нормы невязки от размерности матрицы Гильберта.
	
	\subsection{Постановка задачи.}
	
	Даны невырожденная матрица $A=(a_{ij})\in \mathbb{R}^{n\times n}$ и ненулевой вектор $b\in \mathbb{R}^n$.
	
	Найти вектор $x\in \mathbb{R}^n$, такой что $Ax=b$.
	
	\section{\underline{Алгоритм и условия его применимости.}}
	
	\subsection{Алгоритм решения СЛАУ методом LU-разложения.}
	
	\begin{enumerate}
		\item Ввести матрицу $A=(a_{ij})\in \mathbb{R}^{n\times n}$ и вектор $b=(b_1,...,b_n)^\top\in \mathbb{R}^n$.
		\item Найти такие нижнюю унитреугольную матрицу $L=(l_{ij})\in \mathbb{R}^{n\times n}$ и верхнюю треугольную матрицу $U=(u_{ij})\in \mathbb{R}^{n\times n}$, что $A=LU$. Для этого положить $U=L=0$ -- нулевая матрица и
		для всех $m = 1,2,...,n$:
		\begin{enumerate}
			\item Для всех $j = m,m+1,...,n$ вычислить

			\begin{equation} 
				\label{U_formula}
				\begin{aligned}
					&m=1:u_{1j}=a_{1j}, \\
					&m\neq 1: u_{mj}=a_{mj}-\sum\limits_{k=1}^{m-1}l_{mk}u_{kj}
				\end{aligned}
			\end{equation}
			\item Для всех $i = m,m+1,...,n$ вычислить
			\begin{equation}
				\label{L_formula}
				\begin{aligned}
					&m=1:l_{i1}=\frac{a_{i1}}{u_{11}},\\
					&m\neq 1:l_{im}=\frac{1}{u_{mm}}\Bigg(a_{im}-\sum\limits_{k=1}^{m-1} l_{ik}u_{km}\Bigg)
				\end{aligned}
			\end{equation}
		\end{enumerate}
		\item Найти такой вектор $y=(y_1,...,y_n)^\top \in \mathbb{R}^n$, что $Ly=b$ методом прямой подстановки, то есть для всех $i=1,2,...,n$ вычислить
		\begin{equation}
			\label{y_formula}
			\begin{aligned}
				&i=1:y_1=l_{11}, \\
				&i\neq 1:y_i=b_i-\sum\limits_{j=1}^{i-1}l_{ij}y_j
			\end{aligned}			
		\end{equation}
		\item Найти такой вектор $x=(x_1,...,x_n)^\top \in \mathbb{R}^n$, что $Ux=y$ методом обратной подстановки, то есть для всех $i=n,n-1,...,1$ вычислить
		\begin{equation}
			\label{x_formula}
			\begin{aligned}
				&i=1:x_n=\frac{y_n}{u_{nn}},\\
				&i\neq 1:x_i=\frac{1}{u_{ii}}\Bigg(y_i-\sum\limits_{j=i+1}^{n}u_{ij}x_j\Bigg)
			\end{aligned}
		\end{equation}
		Вектор $x$ - решение СЛАУ $Ax=b$.
	\end{enumerate}

	\subsection{Условия применимости метода LU-разложения.}
	
	\begin{enumerate}
		\item Определитель матрицы А отличен от нуля.
		\item Все ведущие угловые миноры матрицы A отличны от нуля.
	\end{enumerate}

	\section{\underline{Анализ задачи.}}
	
	Для выполнения лабораторной работы была выбрана СЛАУ с матрицей $A\in\mathbb{R}^{10\times10}$ и свободным вектором $b\in\mathbb{R}^{10}$:
	\begin{equation}
		\label{A_matrix}
		A=
		\begin{bmatrix}
			2 & 7 & 5 & 36 & 17 & 38 & 92 & 43 & 87 & 53 \\
			4 & 1 & -14 & 31 & 53 & 23 & 45 & 32 & 16 & 75 \\
			8 & 3 & 1 & 15& 67& 51& 79& 31& 34& 64 \\
			18 & 1 & 39 & 76 & 83 & 31 & 13 & 43 & 57 & 83 \\
			14 & 43 & 12 & 1 & 53 & 57 & 86 & 43 & 24 & 10 \\
			11 & 21 & 35 & 41 & 50 & 62 & 71 & 83 & 92 & 14 \\
			91 & 83 & 74 & 61 & 57 & 49 & 35 & 22 & 14 & 90 \\
			46 & 73 & 12 & 14 & 53 & 67 & 753 & 15 & 61 & 87 \\
			14 & 53 & 68 & 54 & 67 & 97 & 34 & 15 & 52 & 10 \\
			65 & 13 & 53 & 16 & 74 & 80 & 45 & 89 & 14 & 15 \\
		\end{bmatrix},
		b=
		\begin{bmatrix}
			1 \\ 2 \\ 4 \\ 7 \\ 3 \\ 14 \\ 61 \\ 21 \\ 18 \\ 31
		\end{bmatrix}
	\end{equation} 
	
	\subsection{Проверка существования и единственности решения.}
	
	$\det A = 1,8753\cdot 10^{17} \neq 0$ (Вычислено с помощью MATLAB). По теореме Крамера существует и единствен вектор $x\in\mathbb{R}^n:Ax=b$
	
	\section{\underline{Проверка условий применимости метода LU-разложения.}}
	
	Введем обозначения. Пусть дана матрица $K=(k_{ij})\in\mathbb{R}^{n\times n}$:
	\begin{equation*}
		K=
		\begin{bmatrix}
			k_{11} & \dots & k_{1i} & \dots & k_{1n} \\
			\vdots & \ddots & \vdots & \ddots & \vdots \\
			k_{i1} & \dots & k_{ii} & \dots & k_{in} \\
			\vdots & \ddots & \vdots & \ddots & \vdots \\
			k_{n1} & \dots & k_{ni} & \dots & k_{nn}
		\end{bmatrix}
	\end{equation*}
	Матрица
	\begin{equation*}
		K_i=
		\begin{bmatrix}
			k_{11} & \dots & k_{1i} \\
			\vdots & \ddots & \vdots \\
			k_{i1} & \dots & k_{ii} \\
		\end{bmatrix}
	\end{equation*}
	называется ведущей главной подматрицей порядка $i$. Ее определитель $\det K_i$ называется ведущим $i$-ым угловым минором матрицы $K$. Для матрицы $A$ с помощью пакета MATLAB были вычислены ведущие угловые миноры:
	\begin{equation*}
		\begin{aligned}
			&\det A_1 = 2,\\
			&\det A_2 = -26,\\
			&\det A_3 = -706,\\
			&\det A_4 = -1,0440 \cdot 10^5,\\
			&\det A_5 = 1,1441 \cdot 10^7,\\
			&\det A_6 = -5,2863 \cdot 10^8, \\
			&\det A_7 = 3,0467 \cdot 10^{11}, \\
			&\det A_8 = 1,9349 \cdot 10^{14}, \\
			&\det A_9 = -2,1135 \cdot 10^{16}
		\end{aligned}
	\end{equation*}
	Видим, что все ведущие угловые миноры отличны от нуля -- метод LU-разложения применим.
	
	\section{\underline{Тестовый пример.}}
	
	Для демонстрации идеи метода LU-разложения решим СЛАУ $Cx=d$ с матрицей $C=(c_{ij})\in\mathbb{R}^{3\times3}$ и свободным вектором $d=(d_1,...,d_n)\in\mathbb{R}^3$:
	
	\begin{equation*}
		C=
		\begin{bmatrix}
			3 & 4 & 9 \\
			9 & 14 & 33 \\
			15 & 36 & 97
		\end{bmatrix},
		d=
		\begin{bmatrix}
			38 \\ 136 \\ 378
		\end{bmatrix}
	\end{equation*}

	\begin{enumerate}
		\item Найдем нижнюю унитреугольную матрицу $L=(l_{ij})\in\mathbb{R}^{3\times3}$ и верхнюю треугольную матрицу $U=(u_{ij})\in\mathbb{R}^{3\times3}$, такие, что $C=LU$. Для этого:
		\begin{enumerate}
			\item $m=1$ -- ищем первую строку матрицы $U$ и первый столбец матрицы $L$:
			\begin{equation*}
				\begin{aligned}
				&j=1:u_{11}=c_{11}=3,\\
				&j=2:u_{12}=c_{12}=4,\\
				&j=3:u_{13}=c_{13}=9,\\
				&i=1:l_{11}=\frac{c_{11}}{u_{11}}=\frac{3}{3}=1,\\
				&i=2:l_{21}=\frac{c_{21}}{u_{11}}=\frac{9}{3}=3,\\
				&i=3:l_{31}=\frac{c_{31}}{u_{11}}=\frac{15}{3}=5
				\end{aligned}
			\end{equation*}
			\item $m=2$ -- ищем вторую строку матрицы $U$ и второй столбец матрицы $L$:
			\begin{equation*}
				\begin{aligned}
					&j=2:u_{22}=c_{22}-l_{21}u_{12}=14-3\cdot 4=2,\\
					&j=3:u_{23}=c_{23}-l_{21}u_{13}=33-3\cdot 9=6,\\
					&i=2:l_{22}=\frac{c_{22}-l_{21}u_{12}}{u_{22}}=\frac{14-3\cdot 4}{2}=1,\\
					&i=3:l_{32}=\frac{c_{32}-l_{31}u_{12}}{u_{22}}=\frac{36-5\cdot 4}{2}=8
				\end{aligned}
			\end{equation*}
			\item $m=3$ ищем третью строку матрицы $U$ и третий столбец матрицы $L$: 
			\begin{equation*}
				\begin{aligned}
					&j=3:u_{33}=c_{33}-l_{31}u_{13}-l_{32}u_{23}=97-5\cdot 9-8\cdot 6=4,\\
					&i=3:l_{33}=\frac{c_{33}-l_{31}u_{13}-l_{32}u_{23}}{u_{33}}=\frac{97-5\cdot 9-8\cdot 6}{4}=1
				\end{aligned}
			\end{equation*}
			Получили матрицы
			\begin{equation*}
				L=
				\begin{bmatrix}
					1 & 0 & 0\\
					3 & 1 & 0\\
					5 & 8 & 1
				\end{bmatrix},
				U=
				\begin{bmatrix}
					3 & 4 & 9\\
					0 & 2 & 6\\
					0 & 0 & 4
				\end{bmatrix}
			\end{equation*}
		\end{enumerate}
		\item Решаем СЛАУ $Ly=d$ методом прямой подстановки:
		\begin{equation*}
			\begin{aligned}
				&i=1:y_1=d_1=38,\\
				&i=2:y_2=d_2-l_{21}y_1=136-3\cdot 38=22,\\
				&i=3:y_3=d_3-l_{31}y_1-l_{32}y_2=378-5\cdot 38-8\cdot 22=12
			\end{aligned}
		\end{equation*}
		\item Решаем СЛАУ $Ux=y$ методом обратной подстановки:
		\begin{equation*}
			\begin{aligned}
				&i=3:x_3=\frac{y_3}{u_{33}}=\frac{12}{4}=3,\\
				&i=2:x_2=\frac{y_2-u_{23}x_3}{u_{22}}=\frac{22-6\cdot 3}{2}=2,\\
				&i=1:x_1=\frac{y_1-u_{12}x_2-u_{13}x_3}{u_{11}}=\frac{38-4\cdot 2-9\cdot 3}{3}=1
			\end{aligned}
		\end{equation*}
		Получили решение СЛАУ $Cx=d$: $x=(1,2,3)^\top$. Проверка:
		\begin{equation*}
			\begin{bmatrix}
				3 & 4 & 9 \\
				9 & 14 & 33 \\
				15 & 36 & 97
			\end{bmatrix}
			\begin{bmatrix}
				1 \\ 2 \\ 3
			\end{bmatrix}
			=
			\begin{bmatrix}
				38 \\ 136 \\ 378
			\end{bmatrix}
		\end{equation*}
	\end{enumerate}

	\section{\underline{Подготовка контрольных тестов.}}
	
	Программа для решения СЛАУ будет запускаться для СЛАУ $Ax=b$, где матрица $A$ и свободный вектор $b$ определены формулой (\ref{A_matrix}). Для исследования зависимости нормы фактической ошибки решения СЛАУ $H_nx = p^{(n)}$ с матрицей Гильберта $H_n\in\mathbb{R}^{n\times n}$ от размерности матрицы Гильберта и зависимости нормы невязки от размерности матрицы Гильберта свободный вектор $p^{(n)}\in\mathbb{R}^n$ определен по формуле
	\begin{equation}
		\label{hilbert_vector}
		p_i^{(n)}=1-\frac{1}{i+1}, i=1,...,n
	\end{equation}
	
	\section{\underline{Модульная структура программы.}}
	Для проведения исследования была написана программа на языке C++, состоящая из следующих функций:
	\begin{itemize}
		\item zeros(вх.: размерность матрицы $n$, вых.: матрица $A$) -- инициализирует нулевую матрицу $A$.
		\item vector\_add(вх.: векторы $a$ и $b$, вых.: вектор $s$) -- вычисляет сумму $s=a+b$ векторов $a$ и $b$.
		\item vector\_div(вх.: векторы $a$ и $b$, вых.: вектор $d$) -- вычисляет разность $d=a-b$ векторов $a$ и $b$.
		\item matrix\_by\_vector(вх.: матрица $A$, вектор $b$, вых.: вектор $m$) -- вычисляет произведение $m=Ab$ матрицы $A$ на вектор $b$.
		\item LU\_factorization(вх.: матрица $A$, вых.: матрицы $L$ и $U$) -- вычисляет нижнюю унитреугольную матрицу $L$ и верхнюю треугольную матрицу $U$, такие, что $A=LU$.
		\item forward\_substituion(вх.: матрица $A$, вектор $b$, вых.: вектор $x$) -- вычисляет решение $x$ СЛАУ $Ax=b$ с нижней треугольной матрицей $A$ методом прямой подстановки.
		\item backward\_substituion(вх.: матрица $A$, вектор $b$, вых.: вектор $x$) -- вычисляет решение $x$ СЛАУ $Ax=b$ с верхней треугольной матрицей $A$ методом обратной подстановки.
		\item solve\_SLAU(вх.: матрица $A$, вектор $b$, вых.: вектор $x$) -- вычисляет решение $x$ СЛАУ $Ax=b$ методом LU-разложения.
		\item matrix\_print(вх.: матрица $A$) -- печатает в консоль матрицу $A$.
		\item vector\_print(вх.: вектор $b$) -- печатает в консоль вектор $b$.
		\item matrix\_print\_csv(вх.: имя файла \textit{filename}, матрица $A$) -- печатает в файл \textit{filename} матрицу $A$ в формате csv.
		\item vector\_print\_csv(вх.: имя файла \textit{filename}, вектор $b$) -- печатает в файл \textit{filename} вектор $b$ в формате csv.
		\item hilbert(вх.: размерность матрицы $n$, вых.: матрица $H_n$) -- вычисляет матрицу Гильберта $H_n$ размерности $n$.
		\item hilbert\_vector(вх.: размерность вектора $n$, вых.: вектор $p^{(n)}$) -- вычисляет вектор $p^{(n)}$ по формуле (\ref{hilbert_vector}).
		\item hilbert\_expetiment(вх.: имена файлов \textit{filename\_hilbert\_solutions} и \textit{filename\_hilbert\_vectors}, начальная размерность $n_0$, количество шагов увеличения размерности $s$, вых.: векторы $x^{(n)}$) -- вычисляет $s$ раз решение $x^{(i)}$ СЛАУ $H_ix=p^{(i)}$ для всех $i=n_0,n_0+1,...,n_0+s-1$ и печатает $x^{(i)}$ и $p^{(i)}$ в файлы \textit{filename\_hilbert\_solutions} и \textit{filename\_hilbert\_vectors} соответственно в формате csv.
	\end{itemize}
	
\end{document}