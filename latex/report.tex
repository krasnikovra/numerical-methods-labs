\documentclass[a4paper, 12pt]{article}
\usepackage[utf8]{inputenc}
\usepackage[russian]{babel}
\usepackage{amsfonts}
\usepackage{amsmath} 
\usepackage{amssymb}
\usepackage{mathtools}
\usepackage{indentfirst} 

\usepackage{geometry} % Меняем поля страницы
\geometry{left=2cm}% левое поле
\geometry{right=1.5cm}% правое поле
\geometry{top=1.5cm}% верхнее поле
\geometry{bottom=3cm}% нижнее поле

\newtheorem{theorem}{Теорема}

\begin{document}
	% Титульный лист
	\begin{titlepage}
		\begin{center}
			Санкт-Петербургский политехнический университет Петра Великого \\ Физико-механический институт \\ Высшая школа прикладной математики и вычислительной физики
		\end{center}
		\vspace{10em}
		\begin{center}
			\Large Отчет по лабораторной работе №4 \\ по дисциплине "Численные методы"
		\end{center}
		\vspace{1em}
		\begin{center}
			\Huge Решение задачи Коши для ОДУ 1 порядка. Модифицированный метод Эйлера.
		\end{center}
		\vspace{15em}
		{\Large 
			
			Выполнил: студент гр. 5030102/00003 Красников Р.А.
			\vspace{1em}
			
			Преподаватель: Добрецова С.Б.}
		\vspace{\fill}
		\begin{center}
			Санкт-Петербург \\ 2022
		\end{center}
	\end{titlepage}
	\newpage
	
	\section{Формулировка и формализация задачи.}
	
	\subsection{Формулировка задачи.}
	
	Решить задачу Коши для ОДУ 1 порядка, разрешенного относительно производной.
	
	Построить графики точного и полученного решения, а также график ошибки для двух значений шага.
	
	Построить графики и исследовать зависимости фактической точности и числа итераций от заданной точности.
	
	Внести в начальное условие возмущение. Построить график зависимости фактической ошибки от величины возмущения при фиксированной точности.  
	
	Построить график зависимости фактической точности от величины шага. По графику определить порядок точности метода и константу
	
	\subsection{Постановка задачи.}
	
	Дана задача Коши
	\begin{equation} \label{cauchytask}
		\begin{gathered}
			y'=f(x,y), x\in[a,b],\\
			y(a)=y_0
		\end{gathered}
	\end{equation}
	Ее решение -- непрерывно дифференцируемая функция $\varphi$ такая, что
	\begin{equation}
		\begin{gathered}
			\forall x\in[a,b]: \varphi'(x)=f(x,\varphi(x)),\\ 
			\varphi(a)=y_0
		\end{gathered}
	\end{equation}
	Также задана сетка $\{x_i\}_{i=0}^n$.
	
	Найти сеточную функцию $\{(x_i,y_i)\}_{i=0}^n$ такую, что
	\begin{equation}
		y_i\approx\varphi(x_i), i=\overline{0,n}
	\end{equation}

	\section{Алгоритм и условия его применимости.}
	
	\subsection{Алгоритм решения задачи Коши \eqref{cauchytask} модифицированным методом Эйлера.}
	\label{alghorithm_simple}
	
	\begin{enumerate}
		\item Ввести функцию $f(x,y)$, начальное условие $y_0$, отрезок $[a,b]$, число отрезков разбиения $m$ (число точек равномерной сетки составит $m+1$).
		\item Для всех $i=\overline{1,m}$ вычислить
		\begin{equation} \label{method}
			y_i = y_{i-1} + hf(x_{i-1} + \frac{h}{2}, y_{i-1}+\frac{h}{2}f(x_{i-1},y_{i-1}))
		\end{equation}
		где $h=\dfrac{b-a}{m}$, $x_i=a+ih$.
		\item $\{(x_i,y_i)\}_{i=0}^m$ -- искомая сеточная функция
	\end{enumerate}
	
	\textit{Замечание.} Можно задавать не число отрезков разбиения (или число точек равномерной сетки), а непосредственно сам шаг $h$, но в этом случае не гарантируется, что полученная сеточная функция будет определена в точке $b$, то есть формально решение будет найдено для некоторого меньшего отрезка ${[a,x_m] \varsubsetneq [a,b]}$.
	
	\subsection{Алгоритм решения задачи Коши \eqref{cauchytask} модифицированным методом Эйлера с апостериорной оценкой погрешности.} \label{alghorithm_runge}
	
	\begin{enumerate}
		\item Ввести функцию $f(x,y)$, начальное условие $y_0$, отрезок $[a,b]$, число отрезков разбиения $m$ (число точек равномерной сетки составит $m+1$), точность $\epsilon$.
		\item Для всех $i=\overline{1,m}$
		\begin{enumerate}
			\item Положить $n=1$ -- число разбиений на отрезке разбиения $[x_{i-1},x_i]$.
			\item Применить алгоритм \ref{alghorithm_simple}, передав ему на вход функцию $f(x,y)$, начальное условие $y_{i-1}$, отрезок $[x_{i-1}, x_{i}]$, число отрезков разбиения $n$ -- получаем сеточную функцию $\{(\widetilde{x}_j,\widetilde{y}_j)\}_{j=0}^n$.
			\item Положить $y_i^{(n)}=\widetilde{y}_n$.
			\item Применить алгоритм \ref{alghorithm_simple}, передав ему на вход функцию $f(x,y)$, начальное условие $y_{i-1}$, отрезок $[x_{i-1}, x_{i}]$, число отрезков разбиения $2n$ -- получаем сеточную функцию $\{(\widetilde{\widetilde{x}}_j,\widetilde{\widetilde{y}}_j)\}_{j=0}^{2n}$. \label{cycle_step}
			\item Положить $y_i^{(2n)}=\widetilde{\widetilde{y}}_{2n}$.
			\item Если
			\begin{equation}
				|y_i^{(2n)}-y_i^{(n)}| < 3\epsilon,
			\end{equation}
			то положить $y_i=y_i^{(2n)}$ и продолжить вычисления для $i=i+1$ (если $i<m$), иначе положить $n=2n$ и вернуться к шагу \ref{cycle_step}.
		\end{enumerate}
		\item $\{(x_i,y_i)\}_{i=0}^m$ -- искомая сеточная функция, притом такая, что
		\begin{equation}
			|y_i-\varphi(x_i)|<i\epsilon, i=\overline{1,m},
		\end{equation}
		где $\varphi(x)$ -- точное решение.
	\end{enumerate}

	\textit{Замечание.} При кодировании алгоритма 2.2 учтено, что при применении алгоритма 2.1 (п. 2b, 2d) вовсе не обязательно запоминать получаемую сеточную функцию -- достаточно лишь получить последнее значение сеточной функции, применяя рекуррентную формулу \eqref{method}. 

	\subsection{Условия применимости алгоритмов \ref{alghorithm_simple}, \ref{alghorithm_runge} решения задачи Коши \eqref{cauchytask} модифицированным методом Эйлера.}
	
	\begin{enumerate}
		\item Задача Коши \eqref{cauchytask} имеет единственное решение $y=\varphi(x), x\in[a,b]$.
	\end{enumerate}

	Условие применимости в таком виде не удобно в применении на практике. Поэтому можно использовать следующие теоремы:
	\begin{theorem}[О единственности решения задачи Коши] \label{lipschizoncauchyunique} ~\\
		Пусть $\Pi=[a,b]\times[c,d]$, функция $f(x,y) : \Pi \rightarrow \mathbb{R}$ непрерывна в $\Pi$ и удовлетворяет в $\Pi$ условию Липшица по $y$, то есть
		\begin{equation}
			\forall (x,y_1),(x,y_2)\in\Pi : \exists L>0 : |f(x,y_1)-f(x,y_2)|<L|y_1-y_2|,
		\end{equation}
		Если $\varphi_1(x),\varphi_2(x)$ -- решения задачи Коши \eqref{cauchytask}, то $\forall x\in[a,b]:\varphi_1(x)=\varphi_2(x)$.
	\end{theorem}

	\begin{theorem} \label{lipschiz} ~\\
		Если функции $f(x,y)$, $\dfrac{\partial f}{\partial y}(x,y)$ определены и непрерывны в $\Pi=[a,b]\times[c,d]$, то $f(x,y)$ удовлетворяет в $\Pi$ условию Липшица по $y$.
	\end{theorem}
	
	Мы не можем утверждать заранее, какой окажется область значений $[c,d]$ решения задачи Коши $\varphi(x)$. Поэтому в качестве достаточного условия применимости метода можно использовать следующее условие:
	\begin{equation*}
		\forall [c,d]\subset\mathbb{R}, f(x,y),\dfrac{\partial f}{\partial y}(x,y) \text{ определены на } \Pi=[a,b]\times[c,d] : f(x,y), \dfrac{\partial f}{\partial y}(x,y) \text{ непрерывны в } \Pi 
	\end{equation*}
	
	\section{Анализ задачи.}

	Для выполнения работы мне была предложена задача Коши
	\begin{equation} \label{mytaskbad}
		\begin{gathered}
			xy'-2x^2\sqrt{y}=4y, x\in[1,2],\\
			y(1)=0
		\end{gathered}
	\end{equation}
	и ее точное решение $y=x^4\ln^2x$. Однако нетрудно видеть, что $y=0$ также является решением задачи Коши \eqref{mytaskbad} и $\exists x\in[1,2]: x^4\ln^2x\neq0$, то есть решение задачи Коши \eqref{mytaskbad} не единственно.
	
	Поэтому исследование выполнено для задачи Коши
	\begin{equation} \label{mytaskinit}
		\begin{gathered}
			xy'-2x^2\sqrt{y}=4y, x\in[1,2],\\
			y(1)=1
		\end{gathered}
	\end{equation}
	и ее точного решения $y^*=x^4(\ln x+1)^2$. 
	
	\subsection{Проверка существования и единственности решения.}
	\label{solexistance}
	
	Приведем задачу Коши \eqref{mytaskinit} к виду \eqref{cauchytask}
	\begin{equation} \label{mytask}
		\begin{gathered}
			y'=2x\sqrt{y}+4\frac{y}{x}, x\in[1,2],\\
			y(1)=1
		\end{gathered}
	\end{equation}
	Ее точное решение $y^*=x^4(\ln x+1)^2$. Область значений решения -- $[1, 16(\ln2+1)^2]$ В задаче Коши \eqref{mytask} можно считать $f(x,y):[1,2]\times[1, 16(\ln2+1)^2] \rightarrow \mathbb{R}$, поскольку рассмотрения именно такого сужения функции $f(x,y)$ достаточно для исследуемой задачи -- ее решение для суженной функции $f(x,y)$ по прежнему $y^*=x^4(\ln x+1)^2$, а единственность будет показана далее.
	
	Для задачи Коши \eqref{mytask} имеем
	\begin{equation*}
		\begin{gathered}
			f(x,y) = 2x\sqrt{y}+4\frac{y}{x},\\
			\frac{\partial f}{\partial y}(x,y) = \frac{x}{\sqrt{y}} + \frac{4}{x}
		\end{gathered}
	\end{equation*}

	Эти функции непрерывны в любом прямоугольнике $[1,2]\times[1,d], d\geq1$, поэтому если $\varphi(x)$ -- решение задачи Коши \eqref{mytask}, то $\varphi(x)=x^4(\ln x+1)^2$.
	
	\section{Проверка условий применимости метода вычисления определенного интеграла по формуле трех восьмых.}
	
	\begin{enumerate}
		\item Решение задачи Коши \eqref{mytask} единственно. (см. пункт \ref{solexistance}).
	\end{enumerate}

	Условия применимости модифицированного метода Эйлера выполнены.
	
	\section{Тестовый пример.}
	
	Для демонстрации модифицированного метода Эйлера вычислим приближенное значение решения задачи Коши \eqref{mytask} в точке $1,2$ для шагов $0,2$ и $0,1$.
	
	\begin{enumerate}
		\item Шаг $h=0,2$:
			\begin{equation*}
				\begin{aligned}
					x_0&=a=1,\\
					y_0&=y(a)=y(1)=1,\\
					x_1^{(h)}&=x_0+h=1,2,\\
					y_1^{(h)}&=y_0+hf(x_0+\frac{h}{2},y_0+\frac{h}{2}f(x_0,y_0))=\\&=1+0,2f(1+\frac{0,2}{2},1+\frac{0,2}{2}f(1;1))=2,7202
				\end{aligned}
			\end{equation*}	
		\item Шаг $h=0,1$
			\begin{equation*}
				\begin{aligned}
					x_0&=a=1,\\
					y_0&=y(a)=y(1)=1,\\
					x_1^{(\frac{h}{2})}&=x_0+h=1,1,\\
					y_1^{(\frac{h}{2})}&=y_0+hf(x_0+\frac{h}{2},y_0+\frac{h}{2}f(x_0,y_0))=\\&=1+0,1f(1+\frac{0,1}{2},1+\frac{0,1}{2}f(1;1))=1,7347,\\
					x_2^{(\frac{h}{2})}&=x_0+2h=1,2,\\
					y_2^{(\frac{h}{2})}&=y_1^{(\frac{h}{2})}+hf(x_1^{(\frac{h}{2})}+\frac{h}{2},y_1^{(\frac{h}{2})}+\frac{h}{2}f(x_1^{(\frac{h}{2})},y_1^{(\frac{h}{2})}))=\\&=1,7347+0,1f(1,1+\frac{0,1}{2},1,7347+\frac{0,1}{2}f(1,1;1,7347))=2,8389
				\end{aligned}
			\end{equation*}
	\end{enumerate} 

	Оценка погрешности по правилу Рунге:
	\begin{equation*}
		|y_2^{(\frac{h}{2})}-y^*(x_2^{(\frac{h}{2})})|\approx\frac{|y_2^{(\frac{h}{2})}-y_1^{(h)}|}{3}=0,0396
	\end{equation*}
	
	Фактическое значение погрешности:
	\begin{equation*}
		|y_2^{(\frac{h}{2})}-y^*(x_2^{(\frac{h}{2})})|=0,0598
	\end{equation*}

	Погрешность по правилу Рунге оказалась немного меньше, чем фактическая -- это можно объяснить тем, что правило Рунге справделиво в точности при $h\rightarrow0$, а в этом примере $h$ еще достаточно велико. Притом стоит отметить, что по порядку величины значения совпали.
	
	\section{Подготовка контрольных тестов.}
	
	Для построения графиков полученных решений и ошибок были выбраны значения шага $h_1=0,2$ и $h_2=0,1$
	
	Для построения графиков зависимостей фактической ошибки и числа итераций (число разбиений отрезка) от заданной точности были выбраны значения точности $\epsilon_i=10^{-i}, i=\overline{1,11}$, шаг $h=0,1$. 
	
	Для построения графика зависимости фактической ошибки от возмущения начального условия $\Delta y$ были выбраны значения $\Delta y_i=0,01i, i=\overline{0,100}$, шаг $h=0,1$, точность $\epsilon=10^{-5}$.
	
	Для построения графика зависимости фактической точности от величины шага были выбраны значения шага $h_i=2^{-i}, i=\overline{0,14}$
	
\end{document}