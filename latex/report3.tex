\documentclass[a4paper, 12pt]{article}
\usepackage[utf8]{inputenc}
\usepackage[russian]{babel}
\usepackage{amsfonts}
\usepackage{amsmath} 
\usepackage{amsthm}
\theoremstyle{definition}
\usepackage{mathtools}
\usepackage{indentfirst} 

\usepackage{geometry} % Меняем поля страницы
\geometry{left=2cm}% левое поле
\geometry{right=1.5cm}% правое поле
\geometry{top=1.5cm}% верхнее поле
\geometry{bottom=3cm}% нижнее поле

\begin{document}
	% Титульный лист
	\begin{titlepage}
		\begin{center}
			Санкт-Петербургский политехнический университет Петра Великого \\ Физико-механический институт \\ Высшая школа прикладной математики и вычислительной физики
		\end{center}
		\vspace{10em}
		\begin{center}
			\Large Отчет по лабораторной работе №2 \\ по дисциплине "Численные методы"
		\end{center}
		\vspace{1em}
		\begin{center}
			\Huge Решение системы линейных алгебраических уравнений методом Зейделя для схемы Якоби
		\end{center}
		\vspace{15em}
		{\Large 
			
			Выполнил: студент гр. 5030102/00003 Красников Р.А.
			\vspace{1em}
			
			Преподаватель: Добрецова С.Б.}
		\vspace{\fill}
		\begin{center}
			Санкт-Петербург \\ 2021
		\end{center}
	\end{titlepage}
	\newpage
	
	\section{Формулировка и формализация задачи.}
	
	\subsection{Формулировка задачи.}
	
	Решить систему линейных алгебраических уравнений (далее -- СЛАУ) методом Зейделя для схемы Якоби.
	
	Вычислить при фиксированной точности и начальном приближении норму фактической ошибки $\|x-x^*\|$, норму невязки $\|Ax-b\|$ и число итераций.
	
	Построить зависимости нормы фактической ошибки, нормы невязки и числа итераций при фиксированном определителе и начальном приближении от заданной точности.
	
	Построить зависимости нормы фактической ошибки, нормы невязки и числа итераций при фиксированных точности и определителе от начального приближения.
	
	\subsection{Постановка задачи.}
	
	Даны невырожденная матрица $A=(a_{ij})\in \mathbb{R}^{n\times n}$ и вектор $b\in \mathbb{R}^n$.
	
	Найти вектор $x\in \mathbb{R}^n$, такой что $\|x-x^*\|<\epsilon$, где $\epsilon$ -- заданная точность, а $x^*$ -- точное решение СЛАУ $Ax=b$, т.е. $Ax^*\equiv b$.
	
	\section{Алгоритм и условия его применимости.}
	
	\subsection{Алгоритм решения СЛАУ методом Зейделя для схемы Якоби.}
	
	\begin{enumerate}
		\item Ввести матрицу $A=(a_{ij})\in \mathbb{R}^{n\times n}$ и вектор правой части $b\in \mathbb{R}^n$, задать начальное приближение $x^{(0)}$ и точность $\epsilon$, положить $k=1$.
		\item Вычислить матрицу $C=(c_{ij})\in \mathbb{R}^{n\times n}$ и вектор $g\in \mathbb{R}^n$ по формулам
		\begin{equation} \label{C_matrix}
			C=E-D_A^{-1}A=
			\begin{bmatrix}
				0 & -\dfrac{a_{12}}{a_{11}} & \dots & -\dfrac{a_{1n}}{a_{11}} \\[2ex]
				-\dfrac{a_{21}}{a_{22}} & 0 & \dots & -\dfrac{a_{2n}}{a_{22}} \\[2ex]
				-\dfrac{a_{31}}{a_{33}} & -\dfrac{a_{32}}{a_{33}} & \dots & -\dfrac{a_{3n}}{a_{33}} \\
				\vdots & \vdots & \ddots & \vdots \\
				-\dfrac{a_{n1}}{a_{nn}} & -\dfrac{a_{n2}}{a_{nn}} & \dots & 0 \\
			\end{bmatrix}
		\end{equation}
		\begin{equation} \label{g_vector}
			g=-D_A^{-1}b=
			\begin{bmatrix}
				\dfrac{b_1}{a_{11}} \\[2ex]
				\dfrac{b_2}{a_{22}} \\[2ex]
				\dfrac{b_3}{a_{33}} \\
				\vdots \\
				\dfrac{b_n}{a_{nn}}
			\end{bmatrix}
		\end{equation}
		\item Вычислить $x^{(k)}$ по формулам
		\begin{equation} \label{iters}
			\begin{aligned}
				&x^{(k)}_1 = c_{12}x^{(k-1)}_2 + c_{13}x^{(k-1)}_3 + \dots + c_{1n}x^{(k-1)}_n + g_1 \\
				&x^{(k)}_2 = c_{21}x^{(k)}_1 + c_{23}x^{(k-1)}_3 + \dots + c_{2n}x^{(k-1)}_n + g_2 \\
				&x^{(k)}_3 = c_{31}x^{(k)}_1 + c_{32}x^{(k)}_2 + \dots + c_{3n}x^{(k-1)}_n + g_3 \\
				&\dots \\
				&x^{(k)}_n = c_{n1}x^{(k)}_1 + c_{n2}x^{(k)}_2 + \dots + c_{n,n-1}x^{(k)}_{n-1} + g_n \\
			\end{aligned}
		\end{equation}
		\item Если
		\begin{equation} \label{stop_cond}
			\frac{\mu}{1-\mu}\|x^{(k)}-x^{(k-1)}\|_{\infty}<\epsilon
		\end{equation}
		где
		\begin{equation} \label{mu_num}
			\mu=\max\limits_{i=1,2,...,n}\frac{\beta_i}{1-\alpha_i}, \alpha_i=\sum\limits_{j=1}^{i-1}|c_{ij}|, \beta_i=\sum\limits_{j=i}^{n}|c_{ij}|
		\end{equation}
		то положить $x=x^{(k)}$ - решение СЛАУ $Ax=b$ с точностью $\epsilon$ по бесконечной норме, т.е. $\|x-x^*\|_{\infty}<\epsilon$ и остановиться; иначе положить $k=k+1$ и вернуться к шагу 3.
	\end{enumerate}
	
	\subsection{Условия применимости метода Зейделя для схемы Якоби.}
	\newtheorem*{SeidelHardTheorem*}{Теорема о необходимых и достаточных условиях сходимости метода Зейделя}
	\begin{SeidelHardTheorem*}
		Последовательность $\{x^{(k)}\}_{k=0}^\infty$, построенная по методу Зейделя, сходится к точному решению $x^*$ СЛАУ $Ax=b$ тогда и только тогда, когда все собственные числа $|\lambda_B|<1$, где $B=(E-\underline{C}^{-1})\overline{\overline C}$,
		\begin{equation}
			\underline{C} = 
			\begin{bmatrix}
				0 & 0 & 0 & \dots & 0 \\
				c_{21} & 0 & 0 & \dots & 0 \\
				c_{31} & c_{32} & 0 & \dots & 0 \\
				\vdots & \vdots & \vdots & \ddots & \vdots \\
				c_{n1} & c_{n2} & c_{n3} & \dots & 0
			\end{bmatrix} , 
			\overline{\overline C} = 
			\begin{bmatrix}
				c_{11} & c_{12} & c_{13} & \dots & c_{1n} \\
				0 & c_{22} & c_{23} & \dots & c_{2n} \\
				0 & 0 & c_{33} & \dots & c_{3n} \\
				\vdots & \vdots & \vdots & \ddots & \vdots \\
				0 & 0 & 0 & \dots & c_{nn}
			\end{bmatrix}
		\end{equation}
	\end{SeidelHardTheorem*}
	Вычисление собственных чисел матрицы $B$ может оказаться задачей по трудоемкости сравнимой с решаемой, поэтому часто на практике используется более простое достаточное условие сходимости метода Зейделя.
	\newtheorem*{SeidelSoftTheorem*}{Теорема о достаточном условии сходимости метода Зейделя}
	\begin{SeidelSoftTheorem*}
		Если $\|C\|_{1,2,\infty}<1$, то последовательность $\{x^{(k)}\}_{k=0}^\infty$, построенная по методу Зейделя, сходится к точному решению $x^*$ СЛАУ $Ax=b$.
	\end{SeidelSoftTheorem*}
	В данной лабораторной работе СЛАУ $Ax=b$ приводится к виду, удобному для итераций, по схеме Якоби, а значит условие
	\begin{equation}
		\|C\|_{\infty}=\max\limits_{i=1,2,...,n}\sum\limits_{j=1}^n |c_{ij}|=\max\limits_{i=1,2,...,n}(\frac{1}{a_{ii}}\sum\limits_{j=1,j\neq i}^n |a_{ij}|)<1
	\end{equation}
	может быть заменено эквивалентным условием диагонального преобладания
	\begin{equation}
		\sum\limits_{j=1,j\neq i}^n |a_{ij}|<|a_{ii}|, i=1,2,...,n
	\end{equation}
	которое и учитывалось в данной лабораторной работе при построении СЛАУ $Ax=b$.
	
	\section{Анализ задачи.}
	
	Для выполнения лабораторной работы была построена матрица СЛАУ $A\in\mathbb{R}^{10\times10}$ и точное решение $x^*\in\mathbb{R}^{10}$, вектор правой части вычислен по формуле $b\equiv Ax^*$:
	\begin{equation} \label{A_matrix}
		A=
		\begin{bmatrix}
			11,22&0,57&3,15&0,36&0,17&0,38&0,92&0,43&0,87&0,53\\
			0,04&43,31&-0,14&0,31&0,53&3,23&0,45&0,32&0,16&0,75\\
			0,08&0,03&56,41&0,15&0,67&0,51&0,79&0,31&0,34&0,64\\
			0,18&0,01&2,39&46,76&0,83&0,31&0,13&0,43&0,57&0,13\\
			0,14&0,43&0,12&0,01&66,53&0,57&0,86&0,43&0,24&0,10\\
			0,11&0,21&0,35&0,41&0,50&34,62&0,71&0,83&0,92&0,14\\
			0,91&0,83&0,74&0,61&0,57&0,49&36,35&0,22&0,14&0,90\\
			0,46&2,73&0,12&0,14&0,53&0,67&0,53&35,13&0,61&0,87\\
			0,14&0,53&4,68&0,54&0,67&0,97&0,34&0,62&56,12&0,10\\
			0,65&0,13&0,53&0,16&7,74&0,80&0,45&0,89&3,14&15,35
		\end{bmatrix}
	\end{equation}
	\begin{equation}
		\label{x_and_b}
		x^*=
		\begin{bmatrix}
			7\\ 9\\ 3\\ 2\\ 1\\ 5\\ 8\\ 4\\ 6\\ 10
		\end{bmatrix},
		b=
		\begin{bmatrix}
			115,51\\420,29\\189,58\\111,90\\85,65\\194,05\\321,82\\189,43\\369,31\\198,87
		\end{bmatrix}
	\end{equation}

	\subsection{Проверка существования и единственности решения.}
	
	С помощью MATLAB было вычислено $\det A = 3,2079\cdot 10^{15}$, по теореме Крамера существует единственное решение $x^*$ СЛАУ $Ax=b$, именно это решение задано формулой \eqref{x_and_b}.
	
	\subsection{Проверка условий применимости метода Зейделя для схемы Якоби.}
	
	Для анализа применимости метода Зейделя для схемы Якоби для СЛАУ $Ax=b$ в данной лабораторной работе было использовано условие диагонального преобладания для матрицы $A$. Нетрудно видеть, что для матрицы $A$ данное условие выполняется.
	
	\section{Тестовый пример.}
	
	Для демонстрации идеи метода Зейделя для схемы Якоби сделаем 4 итерации метода для СЛАУ $Px=d$ с матрицей $P=(p_{ij})\in\mathbb{R}^{3\times3}$ и свободным вектором $d=(d_1,d_2,d_3)^\top\in\mathbb{R}^3$:
	\begin{equation}
		P=
		\begin{bmatrix}
			4& 1& 2\\ 
			3& -9& -5\\ 
			-2& -1& 7
		\end{bmatrix},
		d=
		\begin{bmatrix}
			7\\-11\\4
		\end{bmatrix}
	\end{equation}
	Точное решение этой СЛАУ:
	\begin{equation}
		x^*=
		\begin{bmatrix}
			1\\1\\1
		\end{bmatrix}
	\end{equation}
	
	\begin{enumerate}
		\item Зададим начальное приближение $x^{(0)}$:
		\begin{equation*}
			x^{(0)}=
			\begin{bmatrix}
				1,7\\0,8\\2,0
			\end{bmatrix}
		\end{equation*}
		\item Вычислим матрицу $C$ и вектор $g$ по формулам \eqref{C_matrix} и \eqref{g_vector}:
		\begin{equation*}
			C=
			\begin{bmatrix}
				0,000  & -0,250 & -0,500\\
				0,333  & 0,000 &  -0,556\\
				0,286 &  0,143  & 0,000\\
			\end{bmatrix},
			g=
			\begin{bmatrix}
				1,750 \\  1,222 \\ 0,571
			\end{bmatrix}
		\end{equation*}
		\item Вычислим число $\mu$ для матрицы $C$ по формуле \eqref{mu_num}:
		\begin{equation}
			\mu=0,833
		\end{equation}
		\item Итерации метода Зейделя:
		\begin{enumerate}
			\item 1-ая итерация:
			\begin{equation*}
				x^{(1)}=
				\begin{bmatrix}
					0,550 \\  0,294  \\ 0,771
				\end{bmatrix}
			\end{equation*}
			\begin{equation*}
				\frac{\mu}{1-\mu}\|x^{(1)}-x^{(0)}\|_{\infty}=6,147
			\end{equation*}
			\item 2-ая итерация:
			\begin{equation*}
				x^{(2)}=
				\begin{bmatrix}
					1,291 \\  1,224  \\ 1,115
				\end{bmatrix}
			\end{equation*}
			\begin{equation*}
				\frac{\mu}{1-\mu}\|x^{(2)}-x^{(1)}\|_{\infty}=4,650
			\end{equation*}
			\item 3-я итерация:
			\begin{equation*}
				x^{(3)}=
				\begin{bmatrix}
					0,886 \\ 0,898 \\ 0,953
				\end{bmatrix}
			\end{equation*}
			\begin{equation*}
				\frac{\mu}{1-\mu}\|x^{(3)}-x^{(2)}\|_{\infty}=2,024
			\end{equation*}
			\item 4-ая итерация:
			\begin{equation*}
				x^{(4)}=
				\begin{bmatrix}
					1,049  \\ 1,042 \\  1,020
				\end{bmatrix}
			\end{equation*}
			\begin{equation*}
				\frac{\mu}{1-\mu}\|x^{(4)}-x^{(3)}\|_{\infty}=0,814
			\end{equation*}
		\end{enumerate}
	\end{enumerate}
	За 4 итерации метода Зейделя получено решение $x=x^{(4)}$ СЛАУ $Px=d$, ошибка составила $\|x-x^*\|_{\infty}=0,049$ по бесконечной норме или $\|x-x^*\|_{2}=0,068$ по второй норме.
	
	\section{Подготовка контрольных тестов.}
	
	Программа для решения СЛАУ методом Зейделя для схемы Якоби будет запускаться для СЛАУ $Ax=b$, где матрица $A$ и вектор $b$ определены по формулам \eqref{A_matrix} и \eqref{x_and_b} соответственно. Точность выбрана $\epsilon=10^{-5}$, а начальным приближением выбран вектор $x^{(0)}$:
	\begin{equation} \label{x0_vector}
		x^{(0)}=
		\begin{bmatrix}
			10\\ 2\\ 4\\ 3\\ 5\\ 6\\ 7\\ 8\\ 9\\ 1
		\end{bmatrix}
	\end{equation}
	
	Для исследования зависимостей нормы фактической ошибки, нормы невязки и числа итераций при фиксированном определителе и начальном приближении от заданной точности программа будет запускаться для начального приближения $x^{(0)}$, определенного по формуле \eqref{x0_vector} и точностей $\epsilon_i=10^{-i}, i=1,2,...,10$.
	
	Для исследования зависимостей нормы фактической ошибки, нормы невязки и числа итераций при фиксированных точности и определителе от начального приближения программа будет запускаться для точности $\epsilon=10^{-8}$ и начальных приближений
	\begin{equation} \label{start_vectors}
		x^{(0),j}=
		\begin{bmatrix}
			x^*_1+R_{1j}\\
			x^*_2+R_{2j}\\
			\vdots \\
			x^*_{10}+R_{10j}\\
		\end{bmatrix},
		j=1,2,...,10
	\end{equation}
	где $R_{ij}\in (-10^{2-j}, 10^{2-j})$ -- случайное возмущение $i$-ой компоненты вектора $x^{(0),j}$ на $j$-ом шаге построения зависимости.
	
	\section{Модульная структура программы.}
	
	Для проведения исследования была написана программа на языке C++, состоящая из следующих функций:
	\begin{itemize}
		\item MatrixPrint(вх.: матрица $A$) -- печатает в консоль матрицу $A$.
		\item VectorPrint(вх.: вектор $x$) -- печатает в консоль вектор $x$.
		\item MatrixPrintCSV(вх.: имя файла file, матрица $A$) -- печатает в файл file матрицу $A$ в формате csv.
		\item VectorSub(вх.: векторы $a$ и $b$, вых.: вектор $s$) -- вычисляет разность $s=a-b$ векторов $a$ и $b$.
		\item MatrixByVector(вх.: матрица $A$, вектор $x$, вых.: вектор $d$) -- вычисляет произведение $d\equiv Ax$ матрицы $A$ на вектор $x$.
		\item VectorNorm(вх.: вектор $x$) -- вычисляет вторую норму $\|x\|_2$ вектора $x$.
		\item VectorNormInf(вх.: вектор $x$) -- вычисляет бесконечную норму $\|x\|_{\infty}$ вектора $x$.
		\item Zeros(вх.: размерность $n$, вых.: матрица $O$) -- генерирует нулевую матрицу $O\in\mathbb{R}^{n\times n}$.
		\item SetJacobiIterationsParameters(вх.: матрица $A$, вектор $b$, вых.: матрица $C$, вектор $g$) -- для СЛАУ $Ax=b$ вычисляет матрицу $C$ и вектор $g$ по формулам \eqref{C_matrix} и \eqref{g_vector}.
		\item GetMuStopConditionNumber(вх.: матрица $C$, вых.: число $\mu$) -- для матрицы $C$ вычисляет число $\mu$, использующееся в условии остановки \eqref{stop_cond}, по формуле \eqref{mu_num}.
		\item SeidelIterations(вх.: матрица $C$, векторы $g$, $x^{(0)}$, точность $\epsilon$, вых.: вектор $x$, число итераций $n$) -- вычисляет рекуррентно вектор $x^{(k)}$ для начального приближения $x^{(0)}$ по формулам \eqref{iters}, пока не выполнится условие остановки \eqref{stop_cond}; как только достигнута точность $\epsilon$, возвращает $x=x^{(k)}$ -- решение СЛАУ с точностью $\epsilon$ по бесконечной норме и число выполненных итераций $n$.
		\item SolveSLAUWithSeidelJacobiMethod(вх.: матрица $A$, векторы $b$, $x^{(0)}$, точность $\epsilon$, вых.: вектор $x$, число итераций $n$) -- вычисляет решение $x$ СЛАУ $Ax=b$ методом Зейделя для схемы Якоби с точностью $\epsilon$ по бесконечной норме для начального приближения $x^{(0)}$, также возвращает число сделанных по методу Зейделя итераций $n$.
		\item FixedStartVectorExperiment(вх.: имя файла file, матрица $A$, векторы $x^*$, $x^{(0)})$, начальная точность $\epsilon_0$, число шагов $s$) -- решает СЛАУ $Ax=b$, где $b\equiv Ax^*$, с точностью ${\epsilon_i=\epsilon_0 \cdot 10^{-i}}$ по бесконечной норме для начального приближения $x^{(0)}$ и записывает в файл file величины $\epsilon_i$, $\|x-x^*\|_{\infty}$, $\|x-x^*\|_{2}$, $\|Ax-b\|_{\infty}$, $\|Ax-b\|_{2}$ в формате csv для всех $i=0,1,...,s-1$.
		\item FixedEpsExperiment(вх.: имя файла file, матрица $A$, вектор $x^*$, точность $\epsilon$, число $m$, число шагов $s$) -- решает СЛАУ $Ax=b$, где $b\equiv Ax^*$, с точностью $\epsilon$ по бесконечной норме для начального приближения $x^{(0),j}$, вычисленного по формуле \eqref{start_vectors}, где ${R_{ij}\in(m\cdot (-10^{-j}), m\cdot 10^{-j})}$, и записывает в файл file величины $\epsilon_i$, $\|x-x^*\|_{\infty}$, $\|x-x^*\|_{2}$, $\|Ax-b\|_{\infty}$, $\|Ax-b\|_{2}$ в формате csv для всех $j=0,1,...,s-1$.
	\end{itemize}
	
\end{document}